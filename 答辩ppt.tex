\documentclass[UTF8]{ctexbeamer}
\usepackage{beamerthemesplit} % 加载主题宏包
\usetheme{Warsaw} % 选用该主题
\usepackage{subfigure}
\usepackage{amsmath,amsthm,amssymb,amsfonts}
\usepackage{amsfonts,booktabs}
\usepackage{amsmath}
\usepackage{lmodern,textcomp}
\usepackage{color}
\usepackage{tikz}
\usepackage{natbib}
\usepackage{multicol}
\usepackage{caption}
\captionsetup[figure]{name={Figure}}
\usepackage{listings}
\usepackage{ctex}
\usepackage{pythonhighlight}

% 用来设置附录中代码的样式

\lstset{
	basicstyle          =   \sffamily,          % 基本代码风格
	keywordstyle        =   \bfseries,          % 关键字风格
	commentstyle        =   \rmfamily\itshape,  % 注释的风格,斜体
	stringstyle         =   \ttfamily,  % 字符串风格
	flexiblecolumns,                % 别问为什么,加上这个
	numbers             =   left,   % 行号的位置在左边
	showspaces          =   false,  % 是否显示空格,显示了有点乱,所以不现实了
	numberstyle         =   \zihao{-5}\ttfamily,    % 行号的样式,小五号,tt等宽字体
	showstringspaces    =   false,
	captionpos          =   t,      % 这段代码的名字所呈现的位置,t指的是top上面
	frame               =   lrtb,   % 显示边框
}

\lstdefinestyle{Python}{
	language        =   Python, % 语言选Python
	basicstyle      =   \zihao{-5}\ttfamily,
	numberstyle     =   \zihao{-5}\ttfamily,
	keywordstyle    =   \color{blue},
	keywordstyle    =   [2] \color{teal},
	stringstyle     =   \color{magenta},
	commentstyle    =   \color{red}\ttfamily,
	breaklines      =   true,   % 自动换行,建议不要写太长的行
	columns         =   fixed,  % 如果不加这一句,字间距就不固定,很丑,必须加
	basewidth       =   0.5em,
}
\title{Summer Study Report}
\author[Xie Yuejin(Advanced Class 2201)]{Xie Yuejin}
\institute[*]{Advanced Class 2201\\
	Huazhong University of Science and Technology\\
	u202210333@hust.edu.cn}
\date{\today}
\begin{document}
\begin{frame}
    \titlepage
\end{frame}

\begin{frame}
	\tableofcontents
\end{frame}

\begin{frame}
\section{Self Introduction}
Xie Yuejin, Advanced Class 2201
\begin{itemize}
	\item Weighted Grades/GPA: 92.17/4.65
	\item Rank: 1/28
	\item "Research" Experience: 
	\begin{itemize}
		\item College Student Entrepreneurship Program(Use Transformer to do channel estimation)
		\item Study for a summer in Professor Wang Bang's lab.
	\end{itemize}
\end{itemize}
\end{frame}

\begin{frame}
\section{First Round of Testing}
This part of the study mainly refers to the book \emph{Dive into Deep Learning} by Mu Li.\\
I learn the basic knowledge and several fundamental models first:
\begin{itemize}
	\item Data Progress, Basci Maths knowledge for AI...
	\item GPU and CPU, How to use Pytorch
	\item Linear Model, Softmax, MLP, SVM, Decision Tree...
\end{itemize}
Just use the knowledge I learned, and successfully passed the first round of testing.
\end{frame}
\section{Second Round of Testing \uppercase\expandafter{\romannumeral1}}
\subsection{前言}
\begin{frame}{前言}	
在最开始的几天里,不太确定自己究竟想选择哪一个方向。在简单了解了几个方向的内容之后,我个人认为NLP和CV两个领域稍微好上手一点,最终选择了NLP方向。\\
这一部分的理论知识学习主要参考了李沐老师的\emph{Dive into Deep Learning},最开始的阻力还是比较大的,基本上听不太懂,了解完基本知识之后开始自己动手实现模型,在边做边学中才逐渐有了深一点的了解.\\
\end{frame}
\subsection{文本预处理}
\begin{frame}[allowframebreaks]{文本预处理}
不同的模型模型预处理不太相同,调用的库函数:
\begin{itemize}
	\item LSTM,TextCNN: Transforms库
	\begin{itemize}
		\item 分词器使用$torchtext.data.get\_ tokenizer("basic\_ english")$ 忽略大小写,特殊字符等
		\item $vocab = build\_vocab\_from\_iterator(reviews\_train, min\_freq=3, specials=['<pad>', '<unk>', '<cls>', '<sep>'])
		vocab.set\_default\_index(vocab['<unk>'])$构建词汇表,并添加特殊字符
		\item $transforms.VocabTransform(vocab=vocab)$将token转化为序号
		\item $transforms.Truncate(max\_seq\_len=max\_len)$截断, $transforms.ToTensor(padding\_value=vocab['<pad>'])$转化成tensor,并填充
		\item $dataset = TensorDataset(text\_transform(reviews), torch.tensor(labels))$返回数据集
	\end{itemize}\newpage
	\item BERT, GPT:
	\begin{itemize}
		\item 专用分词器BertTokenizer, GPT2Tokennizer
		\item 添加特殊字符,如$<sep>,<cls>$等
		\item 其他类似
	\end{itemize}
\end{itemize}
\end{frame}

\subsection{LSTM}
\begin{frame}{LSTM}
LSTM相较RNN,有更多可学习参数,模型更大,下面是LSTM模型图:
\begin{figure}[H] %H为当前位置,!htb为忽略美学标准,htbp为浮动图形
	\centering %图片居中
	\includegraphics[width=1\textwidth]{LSTM.jpg} %插入图片,[]中设置图片大小,{}中是图片文件名
	\caption{1}
\end{figure}
\end{frame}

\begin{frame}[allowframebreaks]{LSTM}
\begin{itemize}
	\item 首先是定义了输入门,遗忘门,输出门,以及候选记忆单元
	
	\begin{align*}
		&\mathbf{I}_{t} =\sigma(\mathbf{X}_t\mathbf{W}_{xi}+\mathbf{H}_{t-1}\mathbf{W}_{hi}+\mathbf{b}_i), \\
		&\mathbf{F}_{t} =\sigma(\mathbf{X}_t\mathbf{W}_{xf}+\mathbf{H}_{t-1}\mathbf{W}_{hf}+\mathbf{b}_f), \\
		&\mathbf{O}_{t} =\sigma(\mathbf{X}_t\mathbf{W}_{xo}+\mathbf{H}_{t-1}\mathbf{W}_{ho}+\mathbf{b}_o),\\
		&\tilde{\mathbf{C}}_t=\tanh(\mathbf{X}_t\mathbf{W}_{xc}+\mathbf{H}_{t-1}\mathbf{W}_{hc}+\mathbf{b}_c),
	\end{align*}
	其中$\sigma(x)$为$sigmoid$激活函数 ,这些计算实际上和RNN中隐状态的计算比较类似\\
	\item 下面我们定义记忆元
	$$
	\mathbf{C}_t=\mathbf{F}_t\odot\mathbf{C}_{t-1}+\mathbf{I}_t\odot\tilde{\mathbf{C}}_t
	$$
	$\mathbf{F}_t\odot\mathbf{C}_{t-1}$实际上表示了遗忘多少先前的记忆元$\mathbf{C}_{t-1}$,而$\mathbf{I}_t\odot\tilde{\mathbf{C}}_t$ 则代表了当前候选记忆元使用的程度
	\item $LSTM$中隐状态的定义如下:
	$$
	\mathbf{H}_t=\mathbf{O}_t\odot\tanh(\mathbf{C}_t)
	$$
	这样的定义可以保证每个元素均在$[-1,1]$之间,防止梯度爆炸
	\item LSTM层的计算基本是这样,我们取最后一层的隐状态加上Dense输出即可
	\item embedding层使用了100维的预训练词向量Glove
\end{itemize}\newpage
代码实现如下(仅给出关键代码):
\begin{figure}[H] %H为当前位置,!htb为忽略美学标准,htbp为浮动图形
	\centering %图片居中
	\includegraphics[scale=0.25]{LSTM2.png} %插入图片,[]中设置图片大小,{}中是图片文件名
	\caption{2}
\end{figure}
结果:
\begin{figure}[H] %H为当前位置,!htb为忽略美学标准,htbp为浮动图形
	\centering %图片居中
	\includegraphics[scale=0.35]{LSTM3.png} %插入图片,[]中设置图片大小,{}中是图片文件名
	\caption{3}
\end{figure}
遇到问题/瑕疵:
\begin{itemize}
	\item 最开始的时候参数矩阵使用了零初始化,导致了梯度消失,损失一直降不下去(改用xavier初始化解决)
	\item 没能实现多层LSTM,以及双向LSTM,使得效果不如框架实现的LSTM
\end{itemize}
\end{frame}

\subsection{TextCNN}
\begin{frame}[allowframebreaks]{TextCNN}
TextCNN的实现总体来说比较轻松,模型图如下:
\begin{figure}[H] %H为当前位置,!htb为忽略美学标准,htbp为浮动图形
	\centering %图片居中
	\includegraphics[scale=0.2]{textcnn_model.jpg} %插入图片,[]中设置图片大小,{}中是图片文件名
	\caption{4}
\end{figure}
\begin{itemize}
	\item 第一层:输入层
	输入层采用了双通道的形式,有两个 $N\times k$ 的输入矩阵,其中一个用预训练好的词嵌入表达,并且frozen;另外一个使用nn.Embedding定义,训练过程发生改变
	\begin{figure}[H] %H为当前位置,!htb为忽略美学标准,htbp为浮动图形
		\centering %图片居中
		\includegraphics[scale=0.4]{textcnn1.png} %插入图片,[]中设置图片大小,{}中是图片文件名
		\caption{5}
	\end{figure}\newpage
	\item 第二层:卷积层
	把$embedded\_size$当作通道数,对每个通道进行一维卷积,通道求和得到输出,可得到多个输出通道,让模型提取不同信息,这里实际上应该是一维卷积,偷个懒使用二维卷积了,效果一样
	\begin{figure}[H] %H为当前位置,!htb为忽略美学标准,htbp为浮动图形
		\centering %图片居中
		\includegraphics[scale=0.4]{textcnn2.png} %插入图片,[]中设置图片大小,{}中是图片文件名
		\caption{6}
	\end{figure}
	\item 第三层:池化连接层	
	将上一步的输出用最大池化,再进行连接,加上dense层输出二分类的概率即可
	\begin{figure}[H] %H为当前位置,!htb为忽略美学标准,htbp为浮动图形
		\centering %图片居中
		\includegraphics[scale=0.6]{textcnn3.png} %插入图片,[]中设置图片大小,{}中是图片文件名
		\caption{7}
	\end{figure}\newpage 
\end{itemize}
结果:
\begin{figure}[H] %H为当前位置,!htb为忽略美学标准,htbp为浮动图形
	\centering %图片居中
	\includegraphics[scale=0.25]{textcnn4.png} %插入图片,[]中设置图片大小,{}中是图片文件名
	\caption{8}
\end{figure}
TextCNN训练速度很快,效果也还不错,总体上没有遇到什么问题
\end{frame}
\section{Second Round of Testing \uppercase\expandafter{\romannumeral2}}
\subsection{BERT从0实现}
\begin{frame}[allowframebreaks]{BERT从0实现}
BERT从0自己实现难度很大,所以主要参考了transformers库中关于BERT的实现
主要实现了以下类:
\begin{itemize}
	\item class BertConfig(object) 包含了BERT模型所需要的大部分参数
	\item class BertEmbeddings(nn.Module) 实际上就是词嵌入+位置编码+$Segment\_Embedding$
	\item class BertAttention(nn.Module) 实际上就是Transformer的Encoder的多头注意力机制
	\item class BertLayer(nn.Module) 整合bertAttention和两个全连接层\newpage
	\item class BertEncoder(nn.Module) BertLayer层堆积
	\item class BertPooler(nn.Module) 对最后一层输出的隐状态全局平均池化,再增加一个线性层映射到二分类
	\item class BertModel(nn.Module) 整合所有部件生成BERT模型类
\end{itemize}
训练结果:
\begin{figure}[H] %H为当前位置,!htb为忽略美学标准,htbp为浮动图形
	\centering %图片居中
	\includegraphics[scale=0.3]{bert.png} %插入图片,[]中设置图片大小,{}中是图片文件名
	\caption{9}
\end{figure}
效果确实比较一般\\\newpage
遇到的问题/瑕疵:
\begin{itemize}
	\item 最开始时损失完全降不下去,后使用学习率调度器使得损失有所下降(应该是模型跑飞了)
	\item 准确率比较差,调不出来一个比较好的参数 
\end{itemize}
\end{frame}
\subsection{Fine-tune:BERT,GPT2}
\begin{frame}[allowframebreaks]{Fine-tune}
	这一部分总体上就比较简单了,调库就行了
	\begin{itemize}
		\item BERT:
		\begin{figure}[H]
			\centering  %图片全局居中
			\subfigure{
				\includegraphics[scale=0.18]{ftbert1}}
			\subfigure{
				\includegraphics[scale=0.18]{ftbert2}}
			\caption{10}	
		\end{figure}
		但是不知道为啥fine-tune效果也不是太好\newpage
		\item GPT2:
		\begin{figure}[H]
			\centering  %图片全局居中
			\subfigure{
				\includegraphics[scale=0.2]{ftgpt}}
			\subfigure{
				\includegraphics[scale=0.2]{ftgpt2}}
			\caption{11}	
		\end{figure}
		GPT2的效果出奇的好,第一个epoch准确率就到了95\%,后面准确率甚至达到了99\%!?\newpage
		\item BERT和GPT的比较
		\begin{itemize}
			\item BERT是Transformer编码器,GPT是Transformer解码器
			\item 预训练差距比较大,BERT是做完形填空,GPT在做预测未来(标准的语言模型),后者显然要更难一些,这可能也是GPT的效果要比BERT差一些的原因之一吧
			\item 目标任务不太一样,BERT主要做的还是主要用于抽取特征(?),而GPT主要还是在做文本生成
		\end{itemize}
	\end{itemize}
\end{frame}
\begin{frame}
	\begin{center}
		\vfill
		\LARGE\bfseries Thank You!
		\vfill
		\normalsize
		Xie Yuejin \\
		\textit{Advanced Class 2201} \\
		\date{}
	\end{center}
\end{frame}
\end{document}
