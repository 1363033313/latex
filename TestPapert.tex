\documentclass[a4paper,11pt,UTF8]{article}
\usepackage{ctex}
\usepackage{amsmath,amsthm,amssymb,amsfonts}
\usepackage{amsmath}
\usepackage[a4paper]{geometry}
\usepackage{graphicx}
\usepackage{microtype}
\usepackage{siunitx}
\usepackage{booktabs}
\usepackage[colorlinks=false, pdfborder={0 0 0}]{hyperref}
\usepackage{cleveref}
\usepackage{esint} 
\usepackage{ctex, draftwatermark, everypage}
\SetWatermarkText{启明学院资料,不要外传}
\SetWatermarkLightness{0.95}
\SetWatermarkScale{0.4}
%opening
\title{启明学院高等数学A$\left(\text{下}\right)$期末试题}
\author{}
\date{}

\begin{document}
\maketitle
期末试题 (1) (150 分钟内完成)\\
一. 填空题 (每小题 4 分, 共 28 分)\\
1. $\displaystyle \sum_{n=1}^{\infty} \frac{n-1}{3^n}=$\\
2. 微分方程 $y^{\prime \prime}-y^{\prime}=x-2$ 的通解为\\
3. 设 $\Sigma$ 为球面 $x^2+y^2+z^2=r^2(r>0)$, 则 $\displaystyle\iint_{\Sigma}\left(x^2+\frac{y^2}{2}+\frac{z^2}{3}\right) d S=$\\
4. 设 $D$ 是以 $(0,0),(1,1),(0,1)$ 为顶点的三角形区域, 则 $\displaystyle \iint_D x^2 e^{y^2} d x d y=$\\
5. 函数 $f(x, y)=x^4+y^4-x^2-2 x y-y^2$ 的极小值为\\
6. 曲线 $x y=4$ 在点 $(2,2)$ 处的曲率为 , 曲率半径为\\
二. 计算题(每小题 7 分, 共 35 分)\\
7. 求幂级数 $\displaystyle \sum_{n=0}^{\infty} \frac{n^2+1}{2^n n !} x^n$ 的和函数(需指明收敛域).\\
8. 求极限 $\displaystyle \lim _{(x, y) \rightarrow(0,0)} \frac{x y}{x+y^2}$. (若极限存在, 求出其值;否则,请阐明理由)\\
9. 设 $\Omega$ 为平面曲线 $\left\{\begin{array}{l}y^2=2 z \\ x=0\end{array}\right.$ 绕 $z$ 轴旋转一周形成的曲面与平面 $z=8$ 所 围的区域. 计算积分 $\displaystyle \iiint_{\Omega}\left(x^2+y^2\right) d x d y d z$.\\
10. 已知 $f(x)=x^2, x \in[-\pi, \pi)$. (1) 求 $f(x)$ 的 Fourier 级数, 并求其和 函数;(2)利用 Parseval 等式求 $\displaystyle \sum_{n=1}^{\infty} \frac{1}{n^4}$ 的和.\\
11. 求曲线 $L:\left\{\begin{array}{l}x^2+y^2+z^2-2 y=4, \\ x+y+z=0\end{array}\right.$ 在点 $(1,1,-2)$ 处的切线和法平面
方程.\\
三. 解答题(每小题 7 分, 共 21 分)\\
12. 讨论函数 $\displaystyle F(\lambda)=\int_0^{+\infty} e^{-\lambda x} \sin x d x$ 在 $(0,+\infty)$ 上的连续性.\\
13. 利用 $\displaystyle \frac{1}{\sin x} \ln \frac{1+a \sin x}{1-a \sin x}=2 \int_0^a \frac{d y}{1-y^2 \sin ^2 x}, \quad(0<a<1)$, 计算积分 $\displaystyle I=\int_0^{\frac{\pi}{2}} \frac{1}{\sin x} \ln \frac{1+a \sin x}{1-a \sin x} d x$\\
(提示: 计算中可能用到 $\displaystyle \frac{1}{\sin ^2 x}=1+\cot ^2 x$ 和 $\displaystyle d(\cot x)=-\frac{d x}{\sin ^2 x}$.)\\
14. 设一个向量与 $O x y, O y z, O z x$ 三坐标面的夹角为 $\varphi, \theta, \omega$, 求 $\cos ^2 \varphi+\cos ^2 \theta+\cos ^2 \omega$ 的值.\\
四. 证明题(共 20 分)\\
15. (7 分) 设函数 $f(x, y)$ 在 $R^2$ 上存在连续偏导数, $\vec{a}_1$ 和 $\vec{a}_2$ 是 $R^2$ 上两个 线性无关的单位向量. 若 $\displaystyle \frac{\partial f}{\partial a_1}(x, y)=\frac{\partial f}{\partial a_1}(x, y)=0$. 证明:在 $R^2$ 上 $f(x, y)$ 为常值函数.\\
16. (7 分)设不含原点的区域 $\Omega$ 有分片光滑封闭曲面 $\Sigma$ 所围成, $\vec{n}$ 为曲 面 $\Sigma$ 的单位外法向量, $\vec{r}=\{x, y, z\}, r=|\vec{r}|$. 证明:
$$
\iiint_{\Omega} \frac{d x d y d z}{r}=\frac{1}{2} \iint_{\Sigma} \cos (\vec{r}, \vec{n}) d S
$$
17. (6 分) 设 $L$ 是逆时针方向的圆周 $(x-a)^2+(y-a)^2=1, f(t)$ 是 $\mathrm{R}$ 上 恒为正值的连续函数. 证明: $\displaystyle \int_L x f(y) d y-\frac{y}{f(x)} d x \geq 2 \pi$.\\
\newpage
期末试题(2)(150分钟内完成)\\
一、填空题 (每空 4 分, 共 28 分)\\
1. 用 Beta 函数表示积分 $\displaystyle \int_0^2(2-x)^{\frac{1}{4}} x^{\frac{3}{4}} d x=$\\
2、级数 $\displaystyle \sum_{n=1}^{\infty} \frac{(-1)^{n-1}}{n}(\ln x)^n$ 的收敛域与和函数分别是$\quad$和$\quad$\\
3、曲面 $z=2 x^2+y^2-x y$ 在点 $(1,1,2)$ 处的法线方程为\\
4、函数 $f(x)=\left\{\begin{array}{cc}0, & -\pi \leq x<0 \\ 1, & 0 \leq x<\pi\end{array}\right.$ 的 Fourier 级数是\\
5、设 $\left(x^{2015}+4 x y^3\right) \mathrm{d} x+\left(a x^2 y^2-2 y^{2066}\right) \mathrm{d} y$ 在整个 $x O y$ 面内是某一函数 $u(x, y)$ 的全微分, 则 $a$ $=$\\
6、设 $L$ 是圆周 $x^2+y^2=1$, 则 $\displaystyle \int_L(x-y)^2 d s=$\\
7、设 $L$ 为曲线 $\left\{\begin{array}{l}x^2+y^2=2 y \\ z=y\end{array}\right.$ 从 轴正向看为顺时针,则 $\displaystyle \oint_L y^2 d x+x y d y+x z d z=$\\
二. 判断题 (每小题 2 分, 共 8 分). 请在正确说法相应的括号中画 “$\checkmark$”, 在错误说法的括号 中画"x".\\
8. 若级数收敛, 则其重排后的级数也必收敛, 其和不变.\\
9. 若函数 $f(x, y)$ 和 $f_y(x, y)$ 都在区域 $[a,+\infty) \times[c, d]$ 上连续, 且 $\displaystyle \int_a^{+\infty} f(x, y) d x$ 关于 $y$ 在 $[c, d]$ 上一致收敛,则 $\displaystyle \frac{d}{d y} \int_a^{+\infty} f(x, y) d x=\int_a^{+\infty} f_y(x, y) d x$.\\
10. 若向量函数 $\vec{F}$ 在区域 $\Omega$ 上有二阶连续偏导数, 则 $\operatorname{div}(\operatorname{rot} \vec{F})=0$.\\
11. 若 $f(x, y)$ 在点 $\left(x_0, y_0\right)$ 沿任意方向的方向导数都存在, 则 $f(x, y)$ 在点 $\left(x_0, y_0\right)$ 可微.\\
三、证明题 (每小题 6 分, 共 12 分)\\
12. 用定义证明 $\displaystyle \lim _{(x, y) \rightarrow(1,0)}\left(x+y^2\right)=1$.\\
13. 设 $\Omega$ 为上半球体 $x^2+y^2+z^2 \leq 1(z \geq 0)$, 函数 $f$ 在 $\Omega$ 上连续. 证明:
$$
\iiint_{\Omega} f(z) d x d y d z=\pi \int_0^1 f(z)\left(1-z^2\right) d z
$$\\
四、计算题 (每小题 7 分, 共 28 分)\\
14. 求函数 $f(x)=x+\ln \left(x+\sqrt{1+x^2}\right)$ 的 Maclaurin 展开式.\\
$
\displaystyle \left(\text { 已知 } \frac{1}{\sqrt{1+x}}=1+\sum_{n=1}^{\infty} \frac{(-1)^n(2 n-1) ! !}{(2 n) ! !} x^n, x \in(-1,1)\right) \text {. }
$\\
15. 设 $D=[0,1] \times[0,1]$, 计算 $\displaystyle \iint_D(x+y) \operatorname{sgn}(x-y) d x d y$.\\
16. 求密度均匀 $(\mu=1)$ 的半球面 $\displaystyle \Sigma: z=\sqrt{1-x^2-y^2}$ 对于 $z$ 轴的转动惯量.\\
17. 计算 $\displaystyle I=\int_L\left(e^x \sin y-2(x+y)\right) \mathrm{d} x+\left(e^x \cos y-x\right) \mathrm{d} y, L$ 是从原点 $O(0,0)$ 沿折线 $y=|x-1|-1$ 至点 $A(2,0)$ 的折线段.\\
五、证明题 (每小题 8 分, 共 24 分)\\
18. 证明 $\displaystyle \sum_{n=1}^{\infty} \frac{\sin (n x)}{\sqrt{n}}\left(1+\frac{1}{n}\right)^n$ 对于 $x$ 在 $(0,2 \pi)$ 内闭一致收敛.\\
19. 已知函数 $z=z(x, y)$ 满足 $\displaystyle x^2 \frac{\partial z}{\partial x}+y^2 \frac{\partial z}{\partial y}=z^2$. 设 $u=x, v=\frac{1}{y}-\frac{1}{x}, \varphi=\frac{1}{z}-\frac{1}{x}$. 证 明: 对函数 $\varphi=\varphi(u, v)$, 成立 $\displaystyle \frac{\partial \varphi}{\partial u}=0$.\\
20. 设 $\Omega$ 为空间二维单连通区域, 三元向量函数 $\vec{F} \in C^1(\Omega)$. 证明: 对 $\Omega$ 内任一闭曲面 $\Sigma$ 都有 $\displaystyle \oiint_{\Sigma} \vec{F} \cdot \vec{n} d S=0$ 的充分必要条件是在 $\Omega$ 内恒有 $\nabla \cdot \vec{F}=0$, 其中 $\vec{n}$ 为 $\Sigma$ 的单位法向量.\\
\newpage
期末试题 (3) (150 分钟内完成)\\
一、填空题 (每空 4 分, 共 28 分)\\
1、设 $\vec{F}=\{\sin x \cos y, \sin y \cos z, \sin z \cos x\}$, 则 rot $\vec{F}=$\\
2、级数 $\displaystyle \sum_{n=1}^{\infty}\left(\ln x+\frac{1}{n}\right)^n$ 的收敛域是\\
3、曲线 $\vec{r}(t)=\{1-\sin t, 1-\cos t, t\}$ 的曲率 $\boldsymbol{K}=$\\
4、函数 $f(x)=\left\{\begin{array}{l}1,0 \leq x \leq h, \\ 0, h<x \leq \pi\end{array}\right.$ 在 $[0, \pi]$ 上的正弦级数是\\
5、交换积分次序后, $\displaystyle \int_{-1}^1 d x \int_{x^3}^1 f(x, y) \mathrm{d} y=$\\
6、函数 $f(x, y)=x^2+y^2+8 x$ 在 $D: x^2+y^2 \leq 16$ 上的最大值是,
最小值是\\
7、直线 $L: x=2 t, y=1, z=t$ 绕 $z$ 轴旋转一周所得的曲面方程是\\
二、判断题 (每小题 2 分, 共 8 分). 请在正确说法相应的括号中画 “$\checkmark$”, 在错误说法 的括号中画 “×”。\\
8. 若级数发散, 则对其任意加括号后所得级数也必发散.\\
9. 若级数 $\displaystyle \sum_{n=1}^{\infty}\left(a_{2 n-1}-a_{2 n}\right)$ 收敛, 且 $\displaystyle \lim _{n \rightarrow \infty} a_n=0$, 则 $\displaystyle \sum_{n=1}^{\infty} a_n$ 收敛.\\
10. 设 $S$ 是球面 $x^2+y^2+z^2=1$ 的外侧, 它关于 $x o y$ 坐标面对称, 所以第二型曲面积分 $\displaystyle \iint_S z^{2017} d x d y=0$\\
11. 若在点 $\left(x_0, y_0\right)$ 的某邻域内 $f(x, y)$ 的两个偏导函数连续, 则 $f(x, y)$ 在点 $\left(x_0, y_0\right)$ 沿任意 方向的方向导数都存在.\\
三、解答题 (每小题 6 分, 共 12 分)\\
12. 计算 $\displaystyle I=\oint_L y d x+z d y+x d z$, 其中 $L$ 是球面 $x^2+y^2+z^2=4 z$ 与平面 $x+z=2$ 的交 线, 从 $z$ 轴正向看去为逆时针方向.\\
13. 设 $D=\left\{(x, y) \mid\left(x^2+y^2\right)^2 \leq 2\left(x^2-y^2\right)\right\}$, 求 $\displaystyle \iint_D(x+y)^2 d x d y$.\\
四、计算题(每小顥 7 分, 共 28 分)\\
14. 求幂级数 $\displaystyle \sum_{n=0}^{\infty} \frac{n+(-1)^n}{(2 n) ! !} x^n$ 的收敛域及和函数.\\
15. 设 $L$ 为圆柱面 $x^2+y^2=1$ 与平面 $z=y$ 的交线, 计算曲线积分 $\displaystyle \int_L z^2 d s$, 并将结果用 $\mathrm{B}$ 函 数表示.\\
16. 设 $S$ 是圆柱面 $x^2+y^2=1$ 介干平面 $z=0$ 和 $z=1$ 之间部分的外侧, 试计算第二型曲面 积分 $\displaystyle I=\iint_S(y-z) x d y d z+(x-y) z d x d y$.\\
17. 计算 $\displaystyle I=\int_L \frac{(y-1) d x-x d y}{x^2+(y-1)^2}$, 其中 $L$ 是椭圆 $x^2+2 y^2=4$, 沿逆时针方向.\\
五、证明题 (每小题 8 分, 共 24 分)\\
18. 证明级数 $\displaystyle \sum_{n=1}^{\infty} \frac{(-1)^{n-1}}{n} \arctan \frac{x}{n}$ 在 $(-\infty, \infty)$ 内一致收敛.\\
19. 设曲面 $S$ 方程由 $F(x, y, z)=0$ 确定, 其中 $F(x, y, z)$ 具有连续的偏导数, 且 $F_z^{\prime} \neq 0$, 又 $S$ 可一对一地投影到 $x O y$ 面的区域 $D$, 证明: $S$ 的面积 $\displaystyle A=\iint_D \frac{\sqrt{F_x^{\prime 2}+F_y^{\prime 2}+F_z^{\prime 2}}}{\left|F_z^{\prime}\right|} d x d y$.\\
20. 设函数 $z=f(x, y)$ 在点 $\left(x_0, y_0\right)$ 的某邻域 $N\left(\left(x_0, y_0\right)\right)$ 内具有二险连续偏导数, 且 $f(x, y)$ 在 $\left(x_0, y_0\right)$ 点取得极大值, 证明: $f_{x x}\left(x_0, y_0\right)+f_{y y}\left(x_0, y_0\right) \leq 0$.\\
\end{document}
