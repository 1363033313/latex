\documentclass[a3paper, 11pt ,UTF8]{article}
\usepackage{ctex}
\usepackage{amsmath,amsthm,amssymb,amsfonts}
\usepackage{amsmath}
\usepackage[a4paper]{geometry}
\usepackage{graphicx}
\usepackage{microtype}
\usepackage{siunitx}
\usepackage{booktabs}
\usepackage[colorlinks=false, pdfborder={0 0 0}]{hyperref}
\usepackage{cleveref}
\usepackage{esint} 
\title{南京大学微积分 II(第一层次)期末试卷 (2022.8.18)\vspace{-2em}}
\date{}

\begin{document}
\maketitle
\noindent 一、(8分) 设 $\displaystyle f(x, y)=\left\{\begin{array}{ll}x y \sin \displaystyle\frac{1}{\sqrt{x^2+y^2}}, & (x, y) \neq(0,0), \\ 0, & (x, y)=(0,0) .\end{array}\right.$ 讨论 $f(x, y)$ 在点 $(0,0)$ 处的连续性、 可偏导性、可微性以及连续可微性.\\
二、计算下列各题 $(7$ 分 $\times 3=21$ 分 $)$\\
1. 求过直线 $L:\left\{\begin{array}{l}10 x+2 y-2 z=27, \\ x+y-z=0\end{array}\right.$ 且与曲面 $3 x^2+y^2-z^2=27$ 相切的平面方程.\\
2. 求旋转抛物面 $x^2+y^2=2 a z(a>0)$ 与半球面 $z=\sqrt{3 a^2-x^2-y^2}$ 所围立体的表面积.\\
3. 计算 $\displaystyle I=\iint_D \frac{1}{x^4+y^2} \mathrm{~d} x \mathrm{~d} y$, 其中 $D: x \geq 1, y \geq x^2$.\\
三、计算下列各题 $(7$ 分 $\times 3=21$ 分 $)$\\
1. 计算 $\displaystyle I=\int_C 2 x \mathrm{~d} x+z \mathrm{~d} y+(x+2 y-z) \mathrm{d} z$, 其中 $C$ 是曲线 $\left\{\begin{array}{l}x^2+y^2+z^2=1, \\ y=z\end{array}\right.$ 上从点 $A(1,0,0)$ 到 $B\left(0, \frac{1}{\sqrt{2}}, \frac{1}{\sqrt{2}}\right)$ 的位于第一卦限的一段曲线.\\
2. 计算 $\displaystyle I=\oint_C \frac{y^2}{2} \mathrm{~d} x-x z \mathrm{~d} y+\frac{y^2}{2} \mathrm{~d} z$, 其中 $C$ 是曲线 $\left\{\begin{array}{l}x^2+y^2+z^2=R^2, \\ x+y=R .\end{array}\right.$ 从 $y$ 轴的正向看去是 依顺时针方向.\\
3. 计算曲面积分 $\displaystyle I=\iint_S\left(x^3+a z^2\right) \mathrm{d} y \mathrm{~d} z+\left(y^3+a x^2\right) \mathrm{d} z \mathrm{~d} x+\left(z^3+a y^2\right) \mathrm{d} x \mathrm{~d} y$, 其中 $S$ 为 $z=$ $\sqrt{a^2-x^2-y^2}$ 的外侧.\\
四、计算下列各题 $(7$ 分 $\times 4=28$ 分 $)$\\
1. 考察级数 $\displaystyle\sum_{n=1}^{\infty}\left(\frac{1}{n}-\arctan \frac{1}{n}\right)$ 的敛散性.\\
2. 判别级数 $\displaystyle\sum_{n=1}^{\infty}(-1)^{n+1} \frac{(2 n-1) ! !}{(2 n) ! !}$ 的敛散性. (提示: $\displaystyle\frac{(2 n-1) ! !}{(2 n) ! !}<\frac{1}{\sqrt{2 n+1}}$ )\\
3. 求 $\displaystyle\sum_{n=0}^{\infty}(n+1)^2 x^n$ 的和函数, 并求数项级数 $\displaystyle\sum_{n=0}^{\infty}(-1)^n(n+1)^2 \frac{1}{3^n}$ 的和.\\
4. 设 $f(x)$ 是周期为 2 的周期函数, 它在 $[-1,1]$ 上的表达式为 $f(x)=x^2$. 将 $f(x)$ 展开成傅里叶级 数, 并求级数 $\displaystyle \sum_{n=1}^{\infty} \frac{(-1)^{n+1}}{n^2}$ 的和.\\
五、计算下列各题 $(7$ 分 $\times 2=14$ 分 $)$\\
1. 求微分方程 $\displaystyle\frac{\mathrm{d} y}{\mathrm{~d} x}=\sin (1+x+y), y(0)=-1$ 的特解.\\
 2. 求微分方程 $\displaystyle\frac{\mathrm{d} y}{\mathrm{~d} x}=\frac{y^3}{2\left(x y^2-x^2\right)}$ 的通解.\\
 六、(8分) 求微分方程 $ \displaystyle y^{\prime \prime}+2 y^{\prime}+y=x \mathrm{e}^{-x}$ 的通解.\\



\end{document}
