\documentclass[a4paper,11pt,UTF8]{article}
\usepackage{ctex}
\usepackage{amsmath,amsthm,amssymb,amsfonts}
\usepackage{amsmath}
\usepackage[a4paper]{geometry}
\usepackage{graphicx}
\usepackage{microtype}
\usepackage{siunitx}
\usepackage{booktabs}
\usepackage[colorlinks=false, pdfborder={0 0 0}]{hyperref}
\usepackage{cleveref}
\usepackage{esint} 
\usepackage{graphicx}
\usepackage{ragged2e}
\usepackage{pifont}
\usepackage{extarrows}
%opening
\title{泾县社会实践日志(其一)}
\author{谢悦晋 \quad 华中科技大学}

\begin{document}

\maketitle
\section{本日调研摘要}
本次实践活动中,我们积极响应当代大学生走向社会,前往安徽省宣城市泾县地区,了解当地文化保护的情况以及旅游业服务业的发展。今日,我们探访了千年前李白写下传世之诗《赠汪伦》的桃花潭,以及被誉为三大古村落之一的查济古村落,我们了解了当地的历史以及民风传统,调研了当地百姓的生活作息方式,通过和当地百姓一对一采访的方式,清楚认识到了村落被开发成旅游景点给当地带来的福祉,也发现了旅游业的一些问题,而等待着我们去解决。


\section{调研实录}
本次调研以实践观察为主,以采访当地居民为辅,下面是调研的两个主要地点:
\subsection{桃花潭}
桃花潭是位于安徽省泾县以西40公里处的历史名胜,是青弋江上游的一段水面。它的水深碧绿,清澈晶莹,翠峦倒映,山光水色,尤显旖旎。唐代诗人李白一曲《赠汪伦》使潭顿时名扬四海,成为历史名胜。景区内自然景观和人文景观融为一体,既有清新秀丽、苍峦叠翠的皖南风光,可观山川之灵气;又有保存完整、风格独特的古代建筑,可发思古之幽情。桃花潭镇不仅仅因诗得名,深厚的文化底蕴和秀美的青山绿水交相辉映,才是它的“内核”所在。


桃花潭景区主要建筑风格为典型的皖南徽派建筑,进入桃花潭景区,映入眼帘的便是水稻,因此我们可以知道这里仍然保留着原始的自己自足的生产方式,跨过麦田,到达文津阁,道路两旁有着五六家售卖扇子饮品等商品的商户,由此可见发展旅游业确实给当地百姓带来了可观的经济收入,在桃花潭镇里穿行,里面已经有了不少商业化的气息,纪念品商店、咖啡馆、土菜馆、奶茶店等等诸如此类,这让我更加坚信了旅游业为当地人民带来的利益。同时,桃花潭边也修建了轮渡和观光竹筏以增加景区收入。


中午我们选择了一家土菜馆,在那里吃了一顿便餐,令人惊奇的是土菜馆价格甚至要比外面的饭馆还要实惠,并没有所谓载客的现象出现,酒饱饭足后我们简单采访了一下酒店老板,了解了当地的一些具体情况,一下是对话的具体实录:
\subsection{查济古村落}

查济古村落位于安徽省泾县桃花潭镇,是中国现存最大的明清古村落之一。查济村原有108座桥梁,108座祠堂、108 座庙宇。现尚有古代建筑140余处。其中桥梁40余座,祠堂30座,庙宇4座,是国家AAAA级景区、全国重点文物保护单位、中国历史文化名村、中华写生第一村、中国传统村落。 查济古村现存古建筑从元至清,且门类众多,有村门、宝塔、牌坊、庙宇、社坛、祠堂、古桥、民居、古井等等,如同古建博物馆。 其中元代建造的“德公厅屋”,位于村中水郎巷,三层门楼,厅内前檐较低,檐柱楠木质,粗矮浑圆,柱础为覆盘式,无雕琢。 明代的“涌清堂”、“进士门”,雕刻细腻,结构精致。

查济古村落与宏村和西递等同为古村落的地方有所区别,它以一种独特的半景区化形式呈现。尽管进入村落的部分带有商业氛围,然而一旦你深入村落,你会发现这里完全没有典型"景点"的痕迹,人们依然保持着原始的生活方式。你可以看到他们在水边洗衣,而且环境保护工作也非常出色。小溪的水依然清澈透明,仍然保持着“潭中鱼可百许头,皆若空游无所依”的境界。整个村落隐藏在深山之中,开发的痕迹并不明显。建筑物大多充满了历史的沧桑,白砖上布满青苔,瓦片也展示着岁月的痕迹。(加当地人的衣食住行)

我们偶遇了一位坐在家门口的老奶奶,在询问之后便坐下来和这位老奶奶聊起了天,以下是对话的具体内容:

\section{总结}
今天所游览的两个名胜之地,共同拥有令人惊叹的特点:绝美的景色和深厚的历史底蕴,仿佛是置身于世外桃源一般。然而,这些美景每天却没有吸引到足够多的游客,或许由于交通不便和宣传不足,每天到访的游客数量并不多,这也让它们保持了一种原始和纯朴的氛围。但是,这样的地方理应被更多人所发现和探寻,因为它们蕴含着无尽的魅力与故事。我们应该思考如何在保留古村落的历史特色和文化遗产的同时,让它们能够适应现代社会的需求和变化,为当地居民和游客带来更多的福祉和乐趣。同时,开发古村落到何种程度,过度开发是否会影响当地居民的生活,也是我们需要思考的问题。

这是一个涉及到数学分析和微积分的问题。我们可以用以下的方式解答这个问题。

首先,我们注意到这个极限的形式很像一个定积分的定义。让我们先看看$\int_0^1 x^k (1-x)^{n-k} dx$,这其实是二项分布的概率密度函数的连续版本,即Beta分布。

Beta分布的概率密度函数为 $f(x; \alpha, \beta) = \frac{1}{B(\alpha, \beta)} x^{\alpha - 1} (1 - x)^{\beta - 1}$,其中 $B(\alpha, \beta)$ 是Beta函数,定义为 $B(\alpha, \beta) = \int_0^1 t^{\alpha - 1} (1 - t)^{\beta - 1} dt = \frac{\Gamma(\alpha)\Gamma(\beta)}{\Gamma(\alpha + \beta)}$。

因此,我们可以将 $\int_0^1 x^k (1-x)^{n-k} dx$ 看作是一个Beta分布的密度函数,其中 $\alpha = k + 1$,$\beta = n - k + 1$。注意,由于我们在计算的是密度函数乘以 $f(x)$ 的积分,而不是密度函数本身的积分,所以我们不需要关心分母中的 $B(\alpha, \beta)$。

这样一来,我们可以将原来的极限改写为
$$
\lim_{n \rightarrow \infty} (n+1) \sum_{k=0}^n \mathbb{E}{X \sim \text{Beta}(k+1, n-k+1)}[f(X)],
$$
其中 $\mathbb{E}{X \sim \text{Beta}(k+1, n-k+1)}[f(X)]$ 是指按照 Beta$(k+1, n-k+1)$ 分布的概率密度函数对 $f(x)$ 取期望值。

当 $n \rightarrow \infty$ 时,由于 $f(x)$ 在 $[0, 1]$ 上连续,根据大数法则,上述极限可以转化为 $f(x)$ 的积分:
$$
\lim_{n \rightarrow \infty} (n+1) \sum_{k=0}^n \mathbb{E}_{X \sim \text{Beta}(k+1, n-k+1)}[f(X)] = \int_0^1 f(x) dx.
$$

这就是我们所求的极限


Question: Let $f$ be continuous on $[0,1]$. Find the limit $\lim {n \rightarrow \infty}(n+1) \sum{k=0}^n \int_0^1 x^k(1-x)^{n-k} f(x) \mathrm{d} x$.

Solution: $\int_0^1 x^k (1-x)^{n-k} dx$ is a Beta function.

The probability density function of the Beta distribution is given by $f(x; \alpha, \beta) = \frac{1}{B(\alpha, \beta)} x^{\alpha - 1} (1 - x)^{\beta - 1}$.

Therefore, $x^k (1-x)^{n-k} = B(\alpha, \beta)\frac{1}{B(\alpha, \beta)} x^{\alpha - 1} (1 - x)^{\beta - 1}$ can be seen as the density function of a Beta distribution multiplied by $B(\alpha, \beta)$, where $\alpha = k + 1$ and $\beta = n - k + 1$.

Hence, we can rewrite the original limit as
$$
\lim_{n \rightarrow \infty} (n+1) \sum_{k=0}^n \int_0^1 x^k(1-x)^{n-k} f(x) \mathrm{d} x \\sim\lim_{n \rightarrow \infty} (n+1) \sum_{k=0}^n \text{Beta}(k+1, n-k+1)\mathbb{E}_{X \sim \text{Beta}(k+1, n-k+1)}[f(X)],
$$
\end{document}
