\documentclass[a4paper,11pt,UTF8]{article}
\usepackage{ctex}
\usepackage{amsmath,amsthm,amssymb,amsfonts}
\usepackage{amsmath}
\usepackage[a4paper]{geometry}
\usepackage{graphicx}
\usepackage{microtype}
\usepackage{siunitx}
\usepackage{booktabs}
\usepackage[colorlinks=false, pdfborder={0 0 0}]{hyperref}
\usepackage{cleveref}

%opening
\title{多元分析学第三章课后习题参考答案}
\author{学数华科xsq}
\setcounter{section}{0}   % 将章节计数器重置为0
\renewcommand{\thesection}{3.\arabic{section}}   % 将“section”标题的编号格式设置为“3.x”
\everymath{\displaystyle}

\begin{document}
第三章课后习题参考答案
% \maketitle

%\begin{abstract}
%  答案仅供参考,请同学们自行思考,错误难免也请指正
%\end{abstract}
\section{n维Euclid空间}
3.1.1\\
证 若 $E^{\prime} \subset E$, 则 $E=E^{\prime} \cup E=E$; 反之, 若 $E=E$, 则由 $E=E^{\prime} \cup E$, 得 到 $E^{\prime} \subset E$.
3.1.2. \\
因为 $E, F \subset \mathbf{R}^n$ 为有界闭集, 所以 $E, F$ 为有界闭集, 于是可知 $E \cap F$ 和 $E \cup$ $F$ 也都是有界闭集.\\
3.1.3.\\
 由于 $F$ 为闭集, 所以 $F^c$ 为开集, 而 $E \backslash F=E \cap F^c$ 也是开集。由于 $E$ 为开集, 所以 $E^c$ 为闭集, 从而 $F \backslash E=F \cap E^c$ 也是闭集.\\
3.1.4\\
(1) $E^{\prime}=\left\{(x, y): x^2+y^2 \geq 3\right\}=\bar{E}, E$ 非闭.
(2) $E^{\prime}=\mathbf{R}^2=\bar{E}, E$ 非闭.
(3) $E^{\prime}=\emptyset, \bar{E}=E, E$ 闭.
(4) $E^{\prime}=\left\{(x, y): y^2-x^2+1 \leq 0\right\}, \bar{E}=E, E$ 闭.
(5) $E^{\prime}=\{(x, y): y=\sin (1 / x), x \in(0,1]$ or $x=0, y \in[-1,1]\} . \bar{E}=E^{\prime}$, E非闭。\\
3.1.5\\
 (1) 由 $\forall k \in\{1,2, \cdots, n\}$, 有
$$
\left|x_k-y_k\right| \leq\left(\sum_{j=1}^n\left|x_j-y_j\right|^p\right)^{1 / p}
$$
得证 $\rho_1(\vec{x}, \vec{y}) \leq \rho_2(\vec{x}, \vec{y})$.
(2) 因为
$$
\left|x_1-y_1\right|^p+\cdots+\left|x_n-y_n\right|^p \leq n \cdot \max _{1 \leq j \leq n}\left|x_j-y_j\right|^p=n \cdot\left(\max _{1 \leq j \leq n}\left|x_j-y_j\right|\right)^p,
$$
所以 $\rho_2(\vec{x}, \vec{y}) \leq n^{1 / p} \rho_1(\vec{x}, \vec{y})$.
3.1.6. \\ 用反证法. 假设 $\forall a \in(0,1)$, 存在 $\vec{x}_a, \vec{y}_a \in A, \vec{x}_a \neq \vec{y}_a$, 使得
$$
\left|F \vec{x}_a-F \vec{y}_a\right| \geq a\left|\vec{x}_a-\vec{y}_a\right|
$$
取 $a=1-\frac{1}{n}$ 时, 相应有 $\vec{x}_n, \vec{y}_n \in A$, 使得
$$
\left|F \vec{x}_n-F \vec{y}_n\right| \geq\left(1-\frac{1}{n}\right)\left|\vec{x}_n-\vec{y}_n\right|
$$
由 $A$ 为 $n$ 维欧氏空间的有界闭集可知, $\left\{\vec{x}_n\right\}$ 和 $\left\{\vec{y}_n\right\}$ 相应有收玫了列 $\left\{\vec{x}_{n_k}\right\}$ 和 $\left\{\vec{y}_{n_k}\right\}$, 记它们的极限依次为 $\vec{x}_0, \vec{y}_0$, 则 $\vec{x}_0 \in A, \vec{y}_0 \in A$. 最后由
$$
\left|F\left(\vec{x}_{n_k}\right)-F\left(\vec{y}_{n_k}\right)\right| \geq\left(1-\frac{1}{n_k}\right)\left|\vec{x}_{n_k}-\vec{y}_{n_k}\right|
$$
和咉射 $F$ 的条件(推出 $F$ 连续), 让 $k \rightarrow \infty$, 得到 $\left|F \vec{x}_0-F \vec{y}_0\right| \geq\left|\vec{x}_0-\vec{y}_0\right|$. 这与已知 矛盾。

\section{多元函数的极限与连续性}
\centerline{$(A)$}
\noindent3.2.1\\
(1)$D = \left\{(x,y)|-1\leqslant\frac{y}{x}\leqslant1\right\}$\\
(2)$D = \left\{(x,y,z)|x^2+y^2>1\right\}$\\
(3)$D = \left\{(x,y)|(2x-x^2-y^2)(x^2+y^2-x)\geqslant0\text{且}x^2+y^2-x\neq 0\right\}$\\
(4)$D = \left\{(x,y)|x^2+y^2>1\right\}$\\
3.2.2\\
由于初等多元函数在定义域内连续,所以平凡地:\\
(1)$\lim_{(x,y) \rightarrow (0,1)}\frac{x+e^y}{x^2+y^2}=e\qquad$(2)$\lim_{(x,y) \rightarrow (2,0)}=\frac{1}{4}$\\
对于(3)(4):\\
$\lim_{(x,y) \rightarrow (0,0)}\frac{\sin xy}{x}=\lim_{(x,y) \rightarrow (0,0)}\frac{\sin xy}{xy}\cdot y = 0$\\
(4)$\lim_{(x,y) \rightarrow (+\infty,+\infty)}(x^2+y^2)e^{-(x+y)}=\lim_{(x,y) \rightarrow (+\infty,+\infty)}\frac{x^2}{e^x}\cdot e^{-y}+\lim_{(x,y) \rightarrow (+\infty,+\infty)}\frac{y^2}{e^y}\cdot e^{-x}=0$\\
3.2.3\\
易知$f(x,y)$在除去点$(0,0)$处均连续,因此考虑$f(x,y)$在$(0,0)$处的连续性:\\
$\lim_{(x,y) \rightarrow (0,0)}\frac{\sin xy}{\sqrt {x^2+y^2}}=\lim_{(x,y) \rightarrow (0,0)}y\cdot \frac{\sin xy}{xy}\cdot\frac{x}{\sqrt{x^2+y^2}}=0=f(0,0)$\\
故$f(x,y)$在$(0,0)$处连续,也即$f(x,y)$在$\mathbf{R}^2$内是连续的\\
3.2.4\\
证:
$\lim_{t \rightarrow 0}f(t\cos\alpha,t\sin\alpha)=\lim_{t\to0}\frac{t\cos^2\alpha\sin\alpha}{t^2\cos^4\alpha+\sin^2\alpha}=0(\frac{t\cos^2\alpha\sin\alpha}{t^2\cos^4\alpha+\sin^2\alpha}<\frac{t\cos^2\alpha}{\sin \alpha}\rightarrow0)$
取$y=x^2,\lim_{(x,y) \rightarrow (0,0)}\frac{x^2y}{x^4+y^2}=\frac{1}{2}\neq f(0,0)$,故$f(x,y)$在$(0,0)$不连续\\
3.2.5\\
$f(x,y)$在$D$内对$x$连续$\Leftrightarrow\forall\varepsilon>0,\exists \delta_1>0,|x-x_1|<\delta_1,|f(x,y)-f(x_1,y)|<\frac{\varepsilon}{2}$\\
$\forall\varepsilon>0,\exists\delta=\min\left\{\delta_1,\frac{\varepsilon}{2L}\right\},|x-x_1|<\delta,|y-y_1|<\delta,\\
|f(x,y)-f(x_1,y_1)|\leqslant|f(x_1,y)-f(x,y)|+|f(x_1,y)-f(x_1,y_1)|<\frac{\varepsilon}{2}+\frac{\varepsilon}{2}=\varepsilon$\\
3.2.6\quad 提示:\\
(1)取$y=kx^2-x$(2)取$y^2=x^2$\\
(3)取$y=kx^3$(4)取$x^3+y^3=kx^4$\\
\centerline{$(B)$}
3.2.1\\
取两点列:$(\sqrt{n+1},\sqrt{n+1}),(\sqrt{n},\sqrt n)$,且$d((\sqrt{n+1},\sqrt{n+1}),(\sqrt{n},\sqrt n))\rightarrow0(n\rightarrow \infty)$\\
由于$f(\sqrt{n+1},\sqrt{n+1})-f(\sqrt{n},\sqrt n)=\frac{1}{2}\nrightarrow0(n\rightarrow\infty)$\\
由$Heine$定理知:$f(x,y)$在$\mathbf{R}^2$上不一致收敛\\
3.2.2\\
(1)$\varepsilon>0,\text{取}\delta=\frac{\varepsilon}{6},|x-1|<\delta,|y-1|<\delta,\\
|x^2+y^2-2|\leqslant|x^2-1|+|y^2-1|=|x+1||x-1|+|x-1||y-1|<3|x-1|+3|y-1|<\varepsilon$
(2)(注本题有误,改为:证$\lim_{(x,y) \rightarrow (0,0)}\frac{1}{\sqrt{xy+1}+1}=\frac{1}{2}$)\\
$\varepsilon>0,\text{取}\delta=\sqrt{\varepsilon},\sqrt{x^2+y^2}<\delta,\\
|\frac{1}{\sqrt{xy+1}+1}-\frac{1}{2}|=|\frac{1-\sqrt{xy+1}}{2(\sqrt{xy+1}+1)}|=|\frac{xy}{2(\sqrt{xy+1}+1)^2}|=\frac{1}{2}|xy|<x^2+y^2<\varepsilon$\\\\
3.2.3\\
不存在,取$y=-\frac{3}{2}x+k\sqrt{|x|}$,该极限不存在是显然的\\
\section{多元函数的偏导数和全微分}
\centerline{$(A)$}
\noindent3.3.1\\ (1) $z_x=2 x y-y^2, z_y=x^2-2 y x$;\\
(2) $z_x=y^{x+1}x^{y-1}+x^yy^x\ln y,z_y=x^{y+1}y^{x-1}+y^xx^y\ln x$;\\
(3) $f_x=y+\frac{1}{y}, f_y=x-\frac{x}{y^2}$;\\
(4) $z_x=\frac{1}{y^2}, z_y=-\frac{2 x}{y^3}$;\\
(5) $z_x=-\frac{2 x \sin x^2}{y}, z_y=-\frac{\cos x^2}{y^2}$;\\
(6) $z_x=\frac{2 x}{y} \sec ^2 \frac{x^2}{y}, z_y=-\frac{x^2}{y^2} \sec ^2 \frac{x^2}{y}$;\\
(7) $z_x=y x^{y-1}, z_y=x^y \ln x$;\\
(8) $g_x=\frac{x}{x^2+y^2}, g_y=\frac{y}{x^2+y^2}$;\\
(9) $u_x=\frac{-2 x}{\left(x^2+y^2+z^2\right)^2}, u_y=\frac{-2 y}{\left(x^2+y^2+z^2\right)^2}, u_x=\frac{-2 z}{\left(x^2+y^2+z^2\right)^2}$;\\
(10) $u_x=y z^{2 y} \ln z, u_y=x z^{2 y} \ln z, u_z=x y z^{2 y-1}$;\\
(11) $u_x=y z(x y)^{x-1}, u_y=x z(x y)^{x-1}, u_z=(x y)^z \ln (x y)$;\\
(12) $u_x=\frac{y}{z} x^{\frac{y}{z}-1}, u_y=\frac{1}{z} x^{\frac{y}{z}} \ln x, u_z=-\frac{y}{z^2} x^{\frac{y}{z}} \ln x$.\\
(13)$z_x=-\frac{|y|}{x^2+y^2},z_y=\frac{xy}{|y|(x^2+y^2)}$\\
(14)$u_x=ye^{\sin yz},u_y =x(e^{\sin yz}+yze^{\sin yz}\cos yz),u_z=xy^2e^{\sin yz}\cos yz$ \\
(15)$u_x=-\frac{y}{x^2}-\frac{1}{z},u_y=\frac{1}{x}-\frac{z}{y^2},u_x=-\frac{1}{y}+\frac{x}{z^2}$\\
3.3.2\\
(1) $z_x(1,0)=z_y(1,0)=z_y(0,1)=0, z_x(0,1)=1$;\\
(2) $z_x\left(0, \frac{\pi}{4}\right)=-1, z_y\left(0, \frac{\pi}{4}\right)=0$;\\
(3) $f_x(1,1,1)=1, f_y(1,1,1)=-1, f_z(1,1,1)=0$;\\
(4) $z_x(0,0)=-1, z_y(0,0)=0$.\\
3.3.3\\
(1) $\mathrm{d} z=-\mathrm{e}^{-x}\cos y \mathrm{~d} x-\mathrm{e}^{-x}\sin y \mathrm{~d} y$;\\
(2) $\mathrm{d} f=\cos (x y)y \mathrm{~d} x+\cos (x y)x \mathrm{~d} y$;\\
(3) $\mathrm{d} g=(2 u+v) \mathrm{d} u+u \mathrm{~d} v$;\\
(4) $\mathrm{d} u=y z x^{yz-1} \mathrm{~d} x+z x^{y z} \ln x \mathrm{~d} y+y x^{y z} \ln x \mathrm{~d} z$;\\
(5) $\mathrm{d} z=-\frac{x y}{\left(x^2+y^2\right)^{3 / 2}} \mathrm{~d} x+\frac{x^2}{\left(x^2+y^2\right)^{3 / 2}} \mathrm{~d} y$;\\
(6) $\mathrm{d} z=\left(2 x y+\frac{1}{y}\right) \mathrm{d} x+\left(x^2-\frac{x}{y^2}\right) \mathrm{d} y$.\\
3.3.4\\
(1) $\mathrm{d} f(1,0)=\mathrm{d} x-\mathrm{d} y$;\\
(2) $\mathrm{d}g\left(2, \frac{\pi}{4}\right)=4 \mathrm{~d} x$;\\
(3) $\mathrm{d} F(100,10)=\frac{G}{5}\left(\frac{1}{20} \mathrm{~d} m-\mathrm{d} r\right)$;\\
(4) $\mathrm{d} f(1,2)=\frac{1}{3} \mathrm{~d} x+\frac{2}{3} \mathrm{~d} y$.\\
3.3.5\\
(1)$f_x(0,0)=\lim_{x\to 0}\frac{f(x,0)-f(0,0)}{x}=\lim_{x\to 0}\sin {\frac{1}{x^2}}$,此极限显然不存在,故$f_x(0,0)$不存在\\
$f_y(0,0)=\lim_{y\to 0}\frac{f(0,0)-f(0,y)}{y}=0$\\
(2)$f_x(0,0)=\lim _{\Delta x \rightarrow 0} \frac{f(0+\Delta x, 0)-f(0,0)}{\Delta x}=\lim _{\Delta x \rightarrow 0} \frac{|\Delta x|}{\Delta x} g(\Delta x, 0)$. 要使 $f_x(0,0)$ 存在,则必然有 $g(0,0)=0$, 此时 $f_x(0,0)=0$.\\
$f_y(0,0)=\lim _{\Delta y \rightarrow 0} \frac{|\Delta y|}{\Delta y} g(0, \Delta y)$, 同理只有当 $g(0,0)=0$ 时$f_y(0,0)$存在, 且 $f_y(0,0)=$ 0 . 故当 $g(0,0)=0$ 时, $f(x, y)$ 可偏导.\\
令 $\rho=\sqrt{x^2+y^2}$, 若 $g(0,0) \neq 0$, 则一定不可微(因为$f(x,y)$不可偏导则一定不可微).\\
而 $g(0,0)=0$ 时, 有:\\
$\frac{f(x,y)-f(0,0)-f_x(0,0)x-f_y(0,0)y }{\rho}
= |x- y| g(x, y)$\\
由于$ \frac{| x- y|}{\rho}$ 有界( $0\leqslant\frac{| x- y|}{\rho} \leqslant \frac{| x|+|a y|}{\rho} \leqslant 2$ ), 又因为 $\lim _{(x,y) \rightarrow (0,0)} g(x,y)=0$.\\
故$\lim_{(x,y)\to (0,0)}\frac{f(x,y)-f(0,0)-f_x(0,0)x-f_y(0,0)y }{\rho}=0$. \\
即当 $g(0,0)=0$ 时 $f(x, y)$ 在 $(0,0)$ 处可微.\\
3.3.6\\
(1) 令 $f(x, y)=(1+x)^m(1+y)^n$.\\
 当 $x, y$ 绝对值很小时.
$
f(x, y)-f(0,0) \approx f_x(0,0)(x-0)+f_y(0,0)(y-0)=m x+n y .
$\\
故 $ f(x, y) \approx f(0,0)+m x+n y=1+m x+n y$.\\
(2) 令 $f(x, y)=\arctan \frac{x+y}{1+x y}$.\\
 当 $|x|,|y|$ 很小时,
$
f(x, y) \approx f(0,0)+f_x(0,0)(x-0)+f_y(0,0)(y-0)=x+y .
$\\
3.3.7\\
(1)
$
f(x, y)=x^y, x=1, y=1, \Delta x=-0.03, \Delta y=0.05 \\
(0.97)^{1.05} \approx f(1,1)+f_x(1,1)\cdot\Delta x+ f_y(1,1)\cdot\Delta y=1.021 ;
$\\
(2)
$
f(x, y)=\sin x \tan y, x=\frac{\pi}{6}, y=\frac{\pi}{4}, \Delta x=-\frac{\pi}{180}, \Delta y=\frac{\pi}{180} \\
\sin 29^{\circ} \tan 46^{\circ} \approx f(\frac{\pi}{6},\frac{\pi}{4})+f_x(\frac{\pi}{6},\frac{\pi}{4})\cdot\Delta x+ f_y(\frac{\pi}{6},\frac{\pi}{4})\cdot\Delta y= 0.502 .
$\\
3.3.8\\
$\frac{\partial g}{\partial x}=-f^{\prime}\left(\frac{1}{r}\right) \frac{1}{r^2} \frac{x}{r}=-\frac{x}{r^3} f^{\prime}\left(\frac{1}{r}\right),$\\ 
$\frac{\partial^2 g}{\partial x^2}=\frac{x^2}{r^6} f^{\prime \prime}\left(\frac{1}{r}\right)-\frac{r^2-3 x^2}{r^5} f^{\prime}\left(\frac{1}{r}\right),$ \\
易知$x,y$等价,由对称性可得: $\frac{\partial^2 g}{\partial y^2}=\frac{y^2}{r^6} f^{\prime \prime}\left(\frac{1}{r}\right)-\frac{r^2-3 y^2}{r^5} f^{\prime}\left(\frac{1}{r}\right)$\\
故 $ \frac{\partial^2 g}{\partial x^2}+\frac{\partial^2 g}{\partial y^2}=\frac{1}{r^4} f^{\prime \prime}\left(\frac{1}{r}\right)+\frac{1}{r^3} f^{\prime}\left(\frac{1}{r}\right) .
$\\
3.3.9\\
(1) $z_{xx}=\mathrm{e}^x(\cos y+x \sin y+2 \sin y),\\
 z_{x y}=\mathrm{e}^x(x \cos y+\cos y-\sin y), z_{yy}=-\mathrm{e}^x(\cos y+x \sin y);$\\
(2) $z_{xx y}=0, z_{x yy}=  -\frac{1}{y^2}$ ;\\
(3)$ z_{xx}=y^4 f_{11}+4 x y^3 f_{12}+4 x^2 y^2 f_{22}+ 2 y f_2,\\ z_{x y}=2 x y^3 f_{11}+5 x^2 y^2 f_{12}+2 x^3 y f_{22}+2 y f_1+2 x f_2, \\
z_{yy}=4 x^2 y^2 f_{11}+4 x^3 y f_{12}+x^4 f_{22}+ 2 x f_1$  ;\\
(4) $u_{xx}=2 f^{\prime}+4 x^2 f^{\prime \prime}, u_{yy}=2 f^{\prime}+4 y^2 f^{\prime \prime}, u_{zz}=2 f^{\prime}+4 z^2 f^{\prime \prime},\\
 u_{x y}=4 x y f^{\prime \prime}, u_{y z}=4 y z f^{\prime \prime}, u_{x z}  =4 x z f^{\prime \prime}$ ;\\
(5)$ z_{xy} =  f_1-\frac{1}{y^2} f_2+x y f_{11}-\frac{x}{y^3} f_{22}-\frac{1}{x^2} g^{\prime}\left(\frac{y}{x}\right)-\frac{y}{x^3} g^{\prime \prime}\left(\frac{y}{x}\right) .$\\
(6)$z_{xx}=\frac{1}{y}f^{\prime\prime}(\frac{x}{y})+\frac{y^2}{x^3}g^{\prime\prime}(\frac{y}{x})\\
z_{xy}=-\frac{x}{y^2}f^{\prime\prime}(\frac{x}{y})-\frac{y}{x^2}g^{\prime}(\frac{y}{x})$\\
3.3.10\\
(1)$ \frac{\partial z}{\partial x}=y \varphi^{\prime}(x y)+\frac{1}{y} \varphi^{\prime}\left(\frac{x}{y}\right) , \frac{\partial z}{\partial y}=x \varphi^{\prime}(x y)-\frac{x}{y^2} \varphi^{\prime}\left(\frac{x}{y}\right) \\ 
dz=\left[y \varphi^{\prime}(x y)+\frac{1}{y} \varphi^{\prime}\left(\frac{x}{y}\right)\right] d x+\left[x \varphi^{\prime}(x y)-\frac{x}{y^2} \varphi^{\prime}\left(\frac{x}{y}\right)\right] dy \\ $
(2) $ \frac{\partial z}{\partial x}=y e^{x y} \sin (x+y)+e^{x y} \cos (x+y) , \frac{\partial z}{\partial y}=x e^{x y} \sin (x+y)+e^{x y} \cos (x+y) \\
 d z=\left[y e^{x y} \sin (x+y)+e^{x y} \cos (x+y)\right] d x+\left[x e^{x y} \sin (x+y)+e^{x y} \cos (x+y)\right] d y$ \\
(3) $\frac{\partial u}{\partial x}=\frac{x}{x^2+y^2+z^2}, \frac{\partial u}{\partial y}=\frac{y}{x^2+y^2+z^2}, \frac{\partial u}{\partial z}=\frac{z}{x^2+y^2+z^2} \\
 d u=\frac{x d x}{x^2+y^2+z^2}+\frac{y d y}{x^2+y^2+z^2}+\frac{z d z}{x^2+y^2+z^2}$ \\ 
(4) $ \frac{\partial u}{\partial x}=2 x f_1^{\prime}+y e^{x y} f_2^{\prime}, \frac{\partial u}{\partial y}=-2 y f_1^{\prime}+x e^{x y} f_2^{\prime}, \frac{\partial u}{\partial z}=f_3^{\prime} \\
 d u=\left[2 x f_1^{\prime}+y e^{x y} f_2^{\prime}\right] d x+\left[-2 y f_1^{\prime}+x e^{x y} f_2^{\prime}\right] d y+f_3^{\prime} d z$ \\
 3.3.11\\
 显然我们要找
 $\left\{\begin{array}{l}
 	\xi=x+a y\\
    \eta=x+b y
 \end{array}\right.$的逆变换
$\left\{\begin{array}{l}
	x=x(\xi,\eta)\\
	y=y(\xi,\eta)
 \end{array}\right.$\\
该存在逆变换的条件是$\frac{\partial(\xi,\eta)}{\partial(x,y)}\neq0$,解得$a\neq b$\\
从而可以解出逆变换:
 $\left\{\begin{array}{l}
	x=\frac{1}{a-b}(a \eta-b \xi)\\
    y=\frac{1}{a-b}(\xi-\eta)
\end{array}\right.$\\
 $
 \frac{\partial u}{\partial \eta}=  \frac{\partial u}{\partial x} \cdot \frac{a}{a-b}+\frac{\partial u}{\partial y} \cdot \frac{-1}{a-b}=\frac{1}{a-b}\left(a \frac{\partial u}{\partial x}-\frac{\partial u}{\partial y}\right) \\
 \frac{\partial^2 u}{\partial \xi \partial \eta}=  \frac{\partial}{\partial \xi}\left(\frac{1}{a-b}\left(a \frac{\partial u}{\partial x}-\frac{\partial u}{\partial y}\right)\right)\\
 =\frac{1}{a-b}\left[a \left(\frac{\partial^2 u}{\partial x^2} \frac{-b}{a-b}+\right.\right. \left.\left.\frac{\partial^2 u}{\partial y \partial x} \frac{1}{a-b}\right)-\left(\frac{\partial^2 u}{\partial x \partial y} \frac{-b}{a-b}+\frac{\partial^2 u}{\partial y^2} \cdot \frac{1}{a-b}\right)\right] \\
 = -\frac{a b}{(a-b)^2}\left[\frac{1}{a b}\frac{\partial^2 u}{\partial x^2}-\frac{a+b}{a b} \frac{\partial^2 u}{\partial x \partial y}+ \frac{\partial^2 u}{\partial y^2}\right]
 $\\\\
 易知
  $\left\{\begin{array}{l}
 	\frac{1}{a b}=3\\
 	-\frac{a+b}{a b}=4
 \end{array}\right.$
解得 $\left\{\begin{array}{l}
	a=-1\\
	b=-\frac{1}{3}
\end{array}\right.$
 或 
$\left\{\begin{array}{l}
    b=-1\\
    a=-\frac{1}{3}
\end{array}\right.$.\\\\
3.3.12\\
$F(1)=f[1, f(1, f(1,1))]=f[1, f(1,1)]=f(1,1)=1$,\\
$
F^{\prime}(1)=  f_1[1, f(1, f(1,1))]+\left.f_2[1, f(1, f(1,1))] \cdot \frac{\mathrm{d} f(x, f(x, x))}{\mathrm{d} x}\right|_{x=1} \\
	=  f_1[1, f(1,1)]+f_2[1, f(1,1)]\left[f_1(1, f(1,1))+ \\
\left.f_2(1, f(1,1)) \frac{\mathrm{d} f(x, x)}{\mathrm{d} x}\right|_{x=1}\right] \\
=  f_1(1,1)+f_2(1,1)\left[f_1(1,1)+f_2(1,1)\left(f_1(1,1)+f_2(1,1)\right)\right] \\
=  a+b[a+b(a+b)] .
$\\
3.3.13\\
证明: 由逆变换定理可以保证存在逆变换:
$\left\{\begin{array}{l}
     x=u\\
     y=\frac{u}{1+u v}
\end{array}\right.$ \\
则 $\varphi=\frac{1}{z}-\frac{1}{x}$ 是 ${u}, {v}$ 的复合函数,故有:\\
$
\frac{\partial \varphi}{\partial u}=-\frac{1}{z^2}\left(\frac{\partial z}{\partial x} \frac{\partial x}{\partial u}+\frac{\partial z}{\partial y} \frac{\partial y}{\partial u}\right)+\frac{1}{u^2}\\
=-\frac{1}{z^2}\left(\frac{\partial z}{\partial x}+\frac{1}{(1+u v)^2} \frac{\partial z}{\partial y}\right)+\frac{1}{u^2} \\
=-\frac{1}{z^2}\left(\frac{\partial z}{\partial x}+\frac{y^2}{x^2} \frac{\partial z}{\partial y}\right)+\frac{1}{u^2} \\
=-\frac{1}{x^2 z^2}\left(x^2 \frac{\partial z}{\partial x}+y^2 \frac{\partial z}{\partial y}\right)+\frac{1}{u^2}\\
=\frac{1}{u^2}-\frac{1}{x^2}=0
$\\
3.3.14\\
不妨设 $\rho=\sqrt{x^2+y^2}$, 则 $\frac{\partial u}{\partial x}=u^{\prime}(\rho) \frac{x}{\rho}, \frac{\partial u}{\partial y}=u^{\prime}(\rho) \frac{y}{\rho}$,\\
$
 \frac{\partial^2 u}{\partial x^2}=u^{\prime \prime}(\rho) \cdot \frac{x^2}{\rho^2}+u^{\prime}(\rho) \cdot \frac{1}{\rho}-u^{\prime}(\rho) \frac{x}{\rho^2} \cdot \frac{x}{\rho}, \\
 \frac{\partial^2 u}{\partial y^2}=u^{\prime \prime}(\rho) \frac{y^2}{\rho^2}+u^{\prime}(\rho) \frac{1}{\rho}-u^{\prime}(\rho) \frac{y}{\rho^2} \cdot \frac{y}{\rho} .
$\\
代入原方程可得:
$u^{\prime \prime}(\rho)+u(\rho)=\rho^2$.\\
对于齐次方程:$u^{\prime \prime}(\rho)+u(\rho)=0$
\\其特征方程为:$\lambda^2+1=0$,解得特征根$\lambda_1=i,\lambda_2=-i$\\
对应齐次解:$u(\rho)=C_1 \cos \rho+C_2 \sin \rho$ \\
对于非齐次方程:$u^{\prime \prime}(\rho)+u(\rho)=\rho^2$,$\alpha=0$非特征根\\
故设其特解为:$u^*=A\rho^2+B\rho+C$,带入原微分方程解得特解:$u^*=\rho^2-2$\\
故$u(\rho)=C_1 \cos \rho+C_2 \sin \rho+\rho^2-2$,\\
也即:$u=u\left(\sqrt{x^2+y^2}\right)=C_1 \cos \sqrt{x^2+y^2}+C_2 \sin \sqrt{x^2+y^2}+x^2+y^2-2$.\\
3.3.15\\
不妨设$\rho=\sqrt{x^2+y^2+z^2}$,
$   \frac{\partial u}{\partial x}=f^{\prime}(\rho) \cdot \frac{x}{\rho}$ \\
$\frac{\partial u^2}{\partial ^2 x}=\frac{f^{\prime}(\rho)}{\rho}+x\left(\frac{f^{\prime \prime}(\rho)}{\rho}-\frac{f^{\prime}(\rho)}{\rho^2}\right) \cdot \frac{x}{\rho}=\frac{f^{\prime}(\rho)}{\rho}+\frac{x^2 f^{\prime \prime}(\rho)}{\rho^2}-\frac{x^2 f^{\prime}(\rho)}{\rho^3}$ \\
易知$x,y,z$三者是相互等价的,所以对称地有: \\
$ \frac{\partial u^2}{\partial ^2 y}=\frac{f^{\prime}(\rho)}{\rho}+\frac{y^2 f^{\prime \prime}(\rho)}{\rho^2}-\frac{y^2 f^{\prime}(\rho)}{\rho^3}, \frac{\partial u^2}{\partial ^2 z}=\frac{f^{\prime}(\rho)}{\rho}+\frac{z^2 f^{\prime \prime}(\rho)}{\rho^2}-\frac{z^2 f^{\prime}(\rho)}{\rho^3}$ \\ 
$\frac{\partial u^2}{\partial ^2 x}+\frac{\partial u^2}{\partial ^2 y}+\frac{\partial u^2}{\partial ^2 z}=f^{\prime \prime}(\rho)+2 \frac{f^{\prime}(\rho)}{\rho}=0 $\\  
$ \frac{f^{\prime \prime}(\rho)}{f^{\prime}(\rho)}=-\frac{2}{\rho}\Rightarrow \ln f^{\prime}(\rho)=\ln \frac{1}{\rho^2}+C=\ln \frac{C_1}{\rho^2}$ \\ 
$f^{\prime}(\rho)=\frac{C_1}{\rho^2}\Rightarrow f(\rho)=-\frac{C_1}{\rho}+C_2$\\
带入初始条件$f^{\prime}(1)=1,f(1)=0$,解得:$f(\rho)=-\frac{1}{\rho}$\\
3.3.16\\
由于 $f(x, y)=h(r)=h\left(\sqrt{x^2+y^2}\right)$, 得
$\frac{\partial f}{\partial x}=h^{\prime}(r) \frac{x}{r}$\\
$\frac{\partial^2 f}{\partial x \partial y}=h^{\prime \prime}(r) \frac{x y}{r^2}-h^{\prime}(r) \frac{x y}{r^3}=0 \Rightarrow h^{\prime \prime}(r)-\frac{1}{r} h^{\prime}(r)=0$\\
此微分方程的处理方法与上题类似,可解得: $h(r)=C_1 r^2+C_2$, \\
即:$f(x, y)=C_1\left(x^2+y^2\right)+C_2$\\
3.3.17\\
$  u=f(x y z) \\$
$ \frac{\partial y}{\partial x}=f^{\prime}(x y z) \cdot y z$ \\  $\frac{\partial^2 u}{\partial x \partial y}=\frac{\partial\left(\frac{\partial u}{\partial x}\right)}{\partial y}=f^{\prime \prime}(x y z) \cdot x z \cdot y z+f^{\prime}(x y z) \cdot z=x y z^2 f^{\prime \prime}(x y z)+z f^{\prime}(x y z)$ \\
$ \frac{\partial ^3 u}{\partial x \partial y \partial z}=\frac{\partial\left(\frac{\partial ^2 u}{\partial x \partial y}\right)}{\partial z}=  2 x y z f^{\prime \prime}(x y z)+x y z^2 \cdot f^{\prime \prime \prime}(x y z) \cdot x y  +1 \cdot f^{\prime}(x y z)+z f^{\prime \prime}(x y z) \cdot x y $\\ 
$=3 x y z f^{\prime \prime}(x y z)+x^2 y^2 z^2 f^{\prime \prime \prime}(x y z) +f^{\prime}(x y z) =x^2 y^2 z^2 f^{\prime \prime \prime}(x y z) $ \\ 
即 $3 x y z f^{\prime \prime}(x y z)+f^{\prime}(x y z)=0$ $\Leftrightarrow$ 设$t =xyz$ ,$3 t f^{\prime \prime}(t)+f^{\prime}(t)=0$ \\ 
该微分方程的处理方法与3.3.15类似,带入初始条件$ f(0)=0, f^{\prime}(1)=1:$\\ 
$\Rightarrow u=\frac{3}{2}(x y z)^{\frac{2}{3}} \\  $
3.3.18\\
记 $ r=\sqrt{x^2+y^2},v = lnr$, 则有
$\frac{\partial u}{\partial x}=\frac{x}{r^2} \cdot f^{\prime}(v), 
 \frac{\partial^2 u}{\partial x^2}=\frac{x^2}{r^4} \cdot f^{\prime \prime}(v)+\frac{1}{r^2} \cdot f^{\prime}(v)-\frac{2 x^2}{r^4} \cdot f^{\prime}(v), $\\
 易知$x,y$是等价的,所以对称地有:$\frac{\partial^2 u}{\partial y^2}=\frac{y^2}{r^4} \cdot f^{\prime \prime}(v)+\frac{1}{r^2} \cdot f^{\prime}(v)-\frac{2 y^2}{r^4} \cdot f^{\prime}(v).$\\
 $ \because\frac{\partial^2 u}{\partial x^2}+\frac{\partial^2 u}{\partial y^2}=\left(x^2+y^2\right)^{\frac{3}{2}}$ \\
 $\Rightarrow f^{\prime \prime}(v)=e^{5v}$\\
$
\Rightarrow f(v)=\frac{1}{25} e^{5 v}+C_1 v+C_2
$
其中 $C_1, C_2$ 是任意常数.\\
3.3.19\\
$\frac{\partial(F,G)}{\partial (u,v)}=x^2-y^2$ 由隐函数定理,在$x^2-y^2\neq 0$的条件下可将$u, v$ 看作 $x, y$ 的隐函数, 两端分别对 $x$ 求导, 得
$
\left\{\begin{array}{l}
	u+x u_x+y v_x=0, \\
	y u_x+v+x v_x=0,
\end{array} \text { 解得 } u_x=\frac{\partial u}{\partial x}=\frac{y v-u x}{x^2-y^2} .\right.
$\\
同理,原方程组两端对 $y$ 求偏导得 $\left\{\begin{array}{l}x u_y+y v_y+v=0, \\ y u_y+x v_y+u=0,\end{array}\right.$ 解得 $v_y=\frac{\partial v}{\partial y}=\frac{y v-u x}{x^2-y^2}$.\\
注:在方程组两边取全微分,相应的解出$\mathrm{d}u,\mathrm{d}v$亦可\\
3.3.20\\
先验证隐函数的存在条件:\\
$J = \frac{\partial(F,G,H)}{\partial (u,v,w)}=$
$\left|\begin{array}{cccc} 
	1 &    1    & 1 \\ 
	v+w &    u+w    & u+v \\ 
	vw&    uw   & uv 
\end{array}\right| 
=(u-v)(u-w)(v-w)\\
$
在$J \neq 0$的条件下可保证隐函数$u(x,y,z),v(x,y,z),z(x,y,z)$的存在性,故在方程两端求全微分,得:\\
$\left\{\begin{array}{l}
	\mathrm{d} u+\mathrm{d} v+\mathrm{d} w=\mathrm{d} x, \\
	(u+w) \mathrm{d} v+(v+w) \mathrm{d} u+(v+u) \mathrm{d} w=\mathrm{d} y \\
	v w \mathrm{~d} u+u w \mathrm{~d} v+u v \mathrm{~d} w=\mathrm{d} z .
\end{array}\right.
$\\
解得
$
\left\{\begin{array}{l}
	\mathrm{d} u=\frac{v-w}{J}\left(u^2 \mathrm{~d} x-u \mathrm{~d} y+\mathrm{d} z\right) \\
	\mathrm{d} v=\frac{u-w}{J}\left(-v^2 \mathrm{~d} x+v \mathrm{~d} y-\mathrm{d} z\right) \\
	\mathrm{d} w=\frac{u-v}{J}\left(w^2 \mathrm{~d} x-w \mathrm{~d} y+\mathrm{d} z\right)
\end{array}\right.
$\\
故
 $\frac{\partial u}{\partial x}=\frac{u^2}{(u-v)(u-w)}, \frac{\partial u}{\partial y}=-\frac{u}{(u-v)(u-w)},
\frac{\partial u}{\partial z}=\frac{1}{(u-v)(u-w)} 
$\\
注:若能记住公式:$\frac{\partial u}{\partial x}=-\frac{1}{J}\frac{\partial(F,G,H)}{\partial (x,v,w)},\frac{\partial u}{\partial y}=-\frac{1}{J}\frac{\partial(F,G,H)}{\partial (y,v,w)}\\
\frac{\partial u}{\partial z}=-\frac{1}{J}\frac{\partial(F,G,H)}{\partial (z,v,w)}$自然更好\\
3.3.21\\
 依题意有, $y$ 是函数, $x , z$ 是自变量。将方程 $z=f(x, y)$ 两边同时对 $x$ 求 导, $0=f_x+f_y \frac{\partial y}{\partial x}$, 则 $\frac{\partial y}{\partial x}=-\frac{f_x}{f_y}$, 于是有:\\
$
\frac{\partial^2 y}{\partial x^2}=\frac{\partial}{\partial x}\left(-\frac{f_x}{f_y}\right)\\
=-\frac{f_y\left(f_{x x}+f_{y x} \frac{\partial y}{\partial x}\right)-f_x\left(f_{y x}+f_{y y} \frac{\partial y}{\partial x}\right)}{f_y^2}
$\\
$
=-\frac{f_y\left(f_{x x}-f_{y x} \frac{f_x}{f_y}\right)-f_x\left(f_{y x}-f_{y y} \frac{f_x}{f_y}\right)}{f_y^2}\\
=-\frac{f_x^2 f_{y y}-2 f_x f_y f_{x y}+f_y^2 f_{y y}}{f_y^3}\\
=\frac{f_xf_{xy}}{f^{2}_y}
$\\
\centerline{$(B)$}
3.3.1\\
证:求函数 $z=x^n f\left(\frac{y}{x^2}\right)$ 的偏导数:\\
$
\begin{aligned}
	& \frac{\partial z}{\partial x}=n x^{n-1} f\left(\frac{y}{x^2}\right)+x^n f^{\prime}(\frac{y}{x^2}) \cdot\left(-\frac{2 y}{x^3}\right)=n x^{n-1} f\left(\frac{y}{x^2}\right)-2 x^{n-3} y f^{\prime}\left(\frac{y}{x^2}\right) \\
	& \frac{\partial z}{\partial y}=x^n f\left(\frac{y}{x^2}\right) \cdot\frac{1}{x^2}=x^{n-2} f^{\prime}\left(\frac{y}{x^2}\right)
\end{aligned}
$\\
所以有:\\
$
\begin{aligned}
	x \frac{\partial z}{\partial x}+2 y \frac{\partial z}{\partial y} & =x\left[n x^{n-1} f\left(\frac{y}{x^2}\right)-2 x^{n-3} y f^{\prime}\left(\frac{y}{x^2}\right)\right]+2 y\left[x^{n-2} f^{\prime}\left(\frac{y}{x^2}\right)\right] \\
	& =n x^n f\left(\frac{y}{x^2}\right)-2 x^{n-2} y f^{\prime}\left(\frac{y}{x^2}\right)+2 x^{n-2} y f^{\prime}\left(\frac{y}{x^2}\right)=n z
\end{aligned}
$\\
3.3.2\\
不妨设$G(x,y,z)=F(z+\frac{1}{x},z-\frac{1}{y})=0$,则由隐函数求导定理,得:\\
$z_x=-\frac{G_x}{G_z}=\frac{F_1\cdot(-\frac{1}{x^2})}{F_1+F_2}=-\frac{1}{2x^2},z_{xx}=\frac{1}{x^3},z_{xy}=0$\\
$z_y=-\frac{G_y}{G_z}=\frac{F_1\cdot\frac{1}{y^2}}{F_1+F_2}=\frac{1}{2y^2},z_{yy}=-\frac{1}{y^3}$\\
$\Rightarrow x^2z_x+y^2z_y=0,x^3z_{xx}+xy(x+y)z_{xy}+y^3z_{yy}=0$\\
3.3.3\\
证明:\\
“$\Rightarrow$”: 由 $f(x, y)$ 是 $n$ 次齐函数知, $\forall t>0$, 有$f(t x, t y)=t^n f(x, y) .$\\
等式两端对 $t$ 求导:
$
x f_1(t x, t y)+y f_2(t x, t y)=n t^{n-1} f(x, y) .
$\\
取 $t=1$ 得:$\quad x f_1(x, y)+y f_2(x, y)=n f(x, y),$\\
即:$\quad x \frac{\partial f}{\partial x}+y \frac{\partial f}{\partial y}=n f(x, y) \text {. }$\\
“$\Leftarrow$” :不妨设 $F(t)=f(t x, t y) (t>0)$,
则有:\\
$\frac{\mathrm{d} F}{\mathrm{~d} t}=x f_1(t x, t y)+y f_2(t x, t y) .$\\
$\Rightarrow t \frac{\mathrm{d} F}{\mathrm{~d} t}=t x f_1(t x, t y)+t y f_2(t x, t y) =n f(t x, t y)=n F(t)$ . \\
即:$\frac{\mathrm{d} F}{F}=\frac{n}{t} \mathrm{~d} t$ .
 $\Rightarrow F(t)=C t^n$, 取 $t=1$, 得 $F(1)=C$.\\
又 $F(t)=f(t x, t y)$, 取 $t=1$, 得 $F(1)=f(x, y)$, 则$C=f(x, y) .$\\
故$f(t x, t y)=t^n f(x, y)$\\
3.3.4\\
不妨设$\frac{f_x}{x}=\frac{f_y}{y}=\frac{f_y}{y}=\lambda$,$\lambda$为常数\\
对$u=f(x,y,z)$取全微分得:\\
$\begin{aligned}
	\mathrm{d}u &=f_x\mathrm{d}x+f_y\mathrm{d}y+f_z\mathrm{d}z
	=\lambda x\mathrm{d}x+\lambda x\mathrm{d}x+\lambda x\mathrm{d}x\\
	&=\frac{\lambda}{2}\mathrm{d}({x^2+y^2+z^2})
	=\frac{\lambda}{2}\mathrm{d}{r^2}=\lambda r\mathrm{d}r
\end{aligned}$\\
$\Rightarrow u=\frac{1}{2}\lambda r^2 + C$,其中$C$为任意常数.
\section{方向导数和梯度}
\centerline{$(A)$}
\noindent3.4.1\quad 易得$\nabla f(P_0) = \left\{3,2,1\right\}$,又由于$f(x,y,z)=x^3y^2z$在$P_0(1,1,1)$处可微,可以使用定理3.4.1,故有:\\
(1)$\frac{\partial f}{\partial \mathbf{l}}\big|_{(1,1,1)}=3$\qquad   (2)$\frac{\partial f}{\partial \mathbf{l}}\big|_{(1,1,1)}=2$\\
(3)$\frac{\partial f}{\partial \mathbf{l}}\big|_{(1,1,1)}=1$\qquad  (4)$\frac{\partial f}{\partial \mathbf{l}}\big|_{(1,1,1)}=2\sqrt{3}$\\
\noindent3.4.2\quad (注:本题疑似有误,因为$f(a,b,c)$的可微性未知,故不可直接使用定理3.4.1)\\
设$\mathbf{l}$上的单位向量为$\left\{x_1,x_2,x_3\right\}$,添上可微性再利用定理3.4.1则有:\\
$\frac{\partial f}{\partial \mathbf{l}}\big|_{(a,b,c)}=\nabla f(a,b,c)\cdot\mathbf{l}=2x_1+3x_2+x_3=0$\\
(1)解方程组
$\begin{cases} 
	2x_1+3x_2+x_3=0 \\
	x_1^2+x_2^2+x_3^2=1
\end{cases}$\\
很容易得到三个解(找到满足第一个方程的解后单位化):\\
$\left\{\frac{1}{\sqrt{5}},0,\frac{-2}{\sqrt{5}}\right\}$,$\left\{0,\frac{1}{\sqrt{10}},\frac{-3}{\sqrt{10}}\right\}$,$\left\{\frac{-3}{\sqrt{13}},\frac{2}{\sqrt{13}},0\right\}$\\
(2)上述方程组显然有无穷组解,故这样的单位向量是无穷多的\\
3.4.3\quad \\
(1)$\nabla f(2,5)=\left\{20,4\right\},\nabla f(3,1)=\left\{6,9\right\}$\\
(2)$\nabla f(1,2)=\left\{\frac{1}{5\sqrt{5}},-\frac{2}{5\sqrt{5}}\right\},\nabla f(3,0)=\left\{-\frac{1}{\sqrt{3}},0\right\}$\\
\\
3.4.4\quad 由梯度定义3.4.2可知:当沿着梯度方向时,函数值增长最快\\
(1)$\mathbf{l}=\nabla f(0,0)=\left\{1,1\right\}$\qquad(2)$\mathbf{l}=\nabla f(2,0,1)=\left\{2,0,1\right\}$\\
(3)$\mathbf{l}=\nabla f(\dfrac{1}{3},\frac{1}{2},\pi)=\left\{-\frac{\pi}{4},-\frac{\pi}{6},-\frac{1}{12}\right\}$\qquad(4)$\mathbf{l}=\nabla f(-1,1)=\left\{-6,8\right\}$\\
3.4.5\quad 同3.4.4,当沿着梯度的反方向时,函数值减小最快\\
(1)$\mathbf{l}=\nabla f(\frac{1}{2},\frac{2}{3})=\left\{\frac{\pi}{3},\frac{\pi}{4}\right\}$\qquad(2)$\mathbf{l}=\nabla f(-1,1,3)=\left\{-\frac{1}{4},-\frac{1}{4},0\right\}$\\
3.4.6 (本题略微超纲,需要利用3.5节的$Lagrange$乘数法)\\$f(x,y,z) = x^2+y^2+z^2$显然可微,可以利用定理3.4.1: \\
$\frac{\partial f}{\partial \mathbf{l}}=\nabla f(x,y,z)\cdot\mathbf{e_l} = \sqrt{2}x-\sqrt{2}y$\\
 即求$g(x,y)=\sqrt{2}(x-y)$ 在条件 $2 x^2+2 y^2+z^2=1$ 下的最大值.
设 $F(x, y, z, \lambda)=\sqrt{2}(x-y)+\lambda\left(2 x^2+2 y^2+z^2-1\right)$, 则由方程组\\
$
\left\{\begin{array}{l}
	\frac{\partial f}{\partial x}=\sqrt{2}+4 \lambda x=0 \\
	\frac{\partial f}{\partial y}=-\sqrt{2}+4 \lambda y=0 \\
	\frac{\partial f}{\partial z}=2 \lambda z=0 \\
	\frac{\partial f}{\partial \lambda}=2 x^2+2 y^2+z^2-1=0
\end{array}\right.
$\\
解得 $z=0$ , $x=-y= \pm \frac{1}{2}$, 故驻点为 $M_1\left(\frac{1}{2},-\frac{1}{2}, 0\right)$ 与 $M_2\left(-\frac{1}{2}, \frac{1}{2}, 0\right)$. 由于最大 值必然存在, 因此只需比较 $\left.\frac{\partial f}{\partial l}\right|_{M_1}=\sqrt{2},\left.\frac{\partial f}{\partial l}\right|_{M_2}=-\sqrt{2}$ 的大小. 所以 $\left.\frac{\partial f}{\partial l}\right|_{M_2}=-\sqrt{2}$ 为所求最大值.\\
3.4.7\quad 由于$f(x,y)$可微,可以利用定理3.4.1:\\
$\frac{\partial f}{\partial \mathbf{l}}=\nabla f(x,y)\cdot\mathbf{e_l} $\\
故有方程组:\\
\renewcommand*{\arraystretch}{2}%段落行距设置}{def}
$
\left\{\begin{array}{l}
	\frac{\partial f}{\partial \mathbf{u}}\big|_p=\frac{3}{5}f_x-\frac{4}{5} f_y=-6\\
	 \frac{\partial f}{\partial \mathbf{v}} \big|_p=\frac{3}{5} f_x+\frac{4}{5} f_y=17 
	
\end{array}\right.
$
\\
$\text {解得}\left.\frac{\partial f}{\partial x}\right|_p=10,\left.\frac{\partial f}{\partial y}\right|_p=15, \\$
故有:$\left.d f\right|_p=10 d x+15 d y .$\\
\centerline{$(B)$}
3.4.1\quad
证: 由于$f(x,y)$在$P_0$处有连续的偏导数,故$f(x,y)$在$P_0$处是可微的,在 $\mathbf{R}^2$ 中利用定理3.4.1,
$$
\begin{aligned}
	\sum_{j=1}^n \frac{\partial f\left(x_0, y_0\right)}{\partial \mathbf{l_j}} & =\sum_{j=1}^n[\nabla f\left(x_0, y_0\right) \cdot\mathbf{l_j}] \\
	& =|\nabla f(x_0,y_0)| \sum_{j=1}^n \cos <\mathbf{l_j}, \nabla f(x_0,y_0)> .
\end{aligned}
$$
不妨设 $\nabla f(x_0,y_0) $与$l_1$  的夹角为 $\alpha$, 则 $l_1, l_2, \cdots, l_n$ 与 $\nabla f(x_0,y_0)$ 的夹角顺次为
$$
\alpha, \alpha+\frac{2 \pi}{n}, \cdots, \alpha+(n-1) \frac{2 \pi}{n} \text {, }
$$
因此
$$
\begin{aligned}
\sum_{j=1}^n \cos <\mathbf{l_j}, \nabla f(x_0,y_0)> & =\cos \alpha+\cos \left(\alpha+\frac{2 \pi}{n}\right)+\cdots+\cos \left[\alpha+(n-1) \frac{2 \pi}{n}\right] \\
	& =\frac{1}{2 \sin \frac{2 \pi}{n}} \sum_{i=1}^{n-1} 2 \cos \left(\alpha+i \frac{2 \pi}{n}\right) \sin \frac{2 \pi}{n} \\
	& =\frac{1}{2 \sin \frac{2 \pi}{n}} \sum_{i=1}^{n-1}\left\{\sin \left[\alpha+(i+1) \frac{2 \pi}{n}\right]-\sin \left[\alpha+(i-1) \frac{2 \pi}{n}\right]\right\} \\
	& =0 .
\end{aligned}\\
$$
证毕\\
3.4.2\quad 依题意知, 温度函数为 $T(x, y)=\frac{k}{\sqrt{x^2+y^2}}$$(k>0)$, 易知沿梯度的反方向温度下降最快,于是有:
$$
\operatorname{grad} T(x, y)=\frac{-k}{\sqrt{\left(x^2+y^2\right)^3}}\{x, y\}, \quad \operatorname{grad} T(3,2)=\frac{-k}{13 \sqrt{13}}\{3,2\},
$$
故蚂蚊应朝方向 $\left\{\frac{3}{\sqrt{13}}, \frac{2}{\sqrt{13}}\right\}$ 爬行, 才能最快达到凉爽处.\\
3.4.3\quad 同上题,沿梯度反方向$z$下降最快,故登山者应沿$\mathbf{l}=-\nabla z(\frac{1}{2},-\frac{1}{2})=\left\{1,-2\right\}$可最快到达山底
\section{多元函数的极值问题}
\centerline{$(A)$}
3.5.2. 求 $f(x, y)=\sin x \sin y$ 在点 $\left(\frac{\pi}{4}, \frac{\pi}{4}\right)$ 的二阶Taylor公式.\\
解 因为 $f_x(x, y)=\cos x \sin y, f_y(x, y)=\sin x \cos y$,\\
$$
f_{x x}(x, y)=-\sin x \sin y, \quad f_{x y}=\cos x \cos y, \quad f_{y y}(x, y)=-\sin x \sin y
$$
所以
$$
f\left(\frac{\pi}{4}, \frac{\pi}{4}\right)=f_x\left(\frac{\pi}{4}, \frac{\pi}{4}\right)=f_y\left(\frac{\pi}{4}, \frac{\pi}{4}\right)=-f_{x x}\left(\frac{\pi}{4}, \frac{\pi}{4}\right)=f_{x y}\left(\frac{\pi}{4}, \frac{\pi}{4}\right)=-f_{y y}\left(\frac{\pi}{4}, \frac{\pi}{4}\right)=\frac{1}{2}
$$
故
$$
f(x, y)=\frac{1}{2}+\frac{1}{2}\left(x-\frac{\pi}{4}\right)+\frac{1}{2}\left(y-\frac{\pi}{4}\right)-\frac{1}{4}\left[\left(x-\frac{\pi}{4}\right)^2-2\left(x-\frac{\pi}{4}\right)\left(y-\frac{\pi}{4}\right)+\left(y-\frac{\pi}{4}\right)^2\right]+o\left(\rho^2\right),
$$\\
其中 $\rho=\sqrt{\left(x-\frac{\pi}{4}\right)^2+\left(y-\frac{\pi}{4}\right)^2}$.\\
3.5.4. 求下列函数的极值: (2) $f(x, y)=e^x\left(x+y^2+2 y\right)$; (4) $f(x, y)=$ $\sin x+\sin y+\sin (x+y), 0<x<\pi, 0<y<\pi$.\\
解 这两个函数的可能极值点只有驻点.\\
(2) 由 $\left\{\begin{array}{l}f_x(x, y)=0 \\ f_y(x, y)=0\end{array}\right.$ 知: $\left\{\begin{array}{l}x+y^2+2 y+1=0 \\ x+y^2+4 y+2=0\end{array}\right.$, 从而得驻点 $P\left(-\frac{1}{4},-\frac{1}{2}\right)$.\\
 经计算得到 $A=f_{x x}(P)=\exp \left(-\frac{1}{4}\right)>0, B=f_{x y}(P)=\exp \left(-\frac{1}{4}\right), C=f_{y y}(P)=$ $3 \exp \left(-\frac{1}{4}\right), \Delta=A C-B^2=2 \exp \left(-\frac{1}{2}\right)>0$. 因此点 $P$ 为函数的极小值点, 极小值 为 $f(P)=-2 \exp \left(-\frac{1}{4}\right)$.\\
(4) 由
$$
\left\{\begin{array}{l}
	f_x(x, y)=\cos x+\cos (x+y)=0 \\
	f_y(x, y)=\cos y+\cos (x+y)=0
\end{array}\right.
$$
解得 $x=y$, 进而得到唯一驻点 $P\left(\frac{\pi}{3}, \frac{\pi}{3}\right)$. 经计算得到 $A=C=-\sqrt{3}<0, B=$ $-\frac{\sqrt{3}}{2}$, 从而有 $\Delta=A C-B^2=3-\frac{3}{4}>0$.\\
 因此点 $P$ 为函数的极大值点, 极大值 为 $f(P)=\frac{3 \sqrt{3}}{2}$.\\
3.5.5. 求下列函数在指定区域 $D$ 上的最大值与最小值:\\
(1) $z=x^2 y(4-x-y), \quad D=\{(x, y) \mid x \geq 0, y \geq 0, x+y \leq 4\}$;
(3) $z=x^2+y^2-12 x+16 y, \quad D=\left\{(x, y) \mid x^2+y^2 \leq 25\right\}$.
解 (1) 函数在区域 $D$ 内仅一个驻点 $P(1,2)$, 得函数值 $f(P)=4$.\\
在两坐标轴上函数恒为 0 . 在直线段 $y=4-x(0 \leq x \leq 4)$ 上, $f(x, 4-x)=$ $0(0 \leq x \leq 4)$. 故, 函数在 $D$ 上的最大值为 4, 最小值为 0 .\\
(3) 由 $\left\{\begin{array}{l}f_x(x, y)=2 x-12=0 \\ f_y(x, y)=2 y+16=0\end{array}\right.$ 解得 $x=6, y=-8$, 点 $(6,-8) \notin D$. 下面考 虑函数在边界 $\partial D$ 的取值情况.\\
$L=x^2+y^2-12 x+16 y+\lambda\left(x^2+y^2-25\right)$ 或 $L=25-12 x+16 y+\mu\left(x^2+y^2-25\right)$. 则由 $\nabla L=\overrightarrow{0}$, 得到 $\lambda+1= \pm 2$ (或 $\mu= \pm 2$ ), $x=\frac{6}{\lambda+1}= \pm 3$ (或 $x=\frac{6}{\mu}= \pm 3$ ), 同 时 $y=-\frac{8}{\lambda+1}=\mp 4$ (或 $\left.y=-\frac{8}{\mu}=\mp 4\right)$. 因为 $f(3,-4)=-75, f(-3,4)=125$, 所 以函数在 $D$ 上的最大值为 125, 最小值为 -75 .\\
3.5 .6 求原点到曲线 $\left\{\begin{array}{l}x^2+y^2=z, \\ x+y+z=1\end{array}\right.$ 的最长和最短距离.\\
解 设目标函数 $d^2=f(x, y, z)=x^2+y^2+z^2$, 令
$L=x^2+y^2+z^2+\lambda\left(x^2+y^2-z\right)+\mu(x+y+z-1)$. \\
由 $\left\{\begin{array}{l}L_x=2 x+2 \lambda x+\mu=0, 
	\\ L_y=2 y+2 \lambda y+\mu=0, \text { 的前两式相减, 得到 }(\lambda+1)(x-y)=0 
	\\ L_z=2 z-\lambda+\mu=0\end{array}\right.$\\
$ \text {由于 } \lambda \neq-1$ (否则有 $\mu=0, z=-1 / 2<0$, 不合约束条件), 所以 $x=y$.\\
 再联立约束条 件 $z=x^2+y^2$ 与 $x+y+z=1$,\\
  可解出 $x=y=\frac{1}{2}(-1 \pm \sqrt{3}), z=2 x^2=2 \mp \sqrt{3}$. \\
 于是得到 $d^2=9 \mp 5 \sqrt{3}$. 故, 最长距离 $=\sqrt{9+5 \sqrt{3}}$, 最短距离 $=\sqrt{9-5 \sqrt{3}}$.\\
注: 可由 $\nabla f, \nabla g, \nabla h$ 的混合积 $2(2 z+1)(x-y)=0$ 得到 $x=y$, \\
其中 $g(x, y, x)=$ $x^2+y^2-z, h(x, y, z)=x+y+z-1$.\\
3.5.9. 求函数 $f(x, y, z)=x+2 y+3 z$ 在圆柱 $x^2+y^2=2$ 与平面 $y+z=1$ 的交 线椭圆上的最大值与最小值.\\
解 目标函数为 $f(x, y, z)=x+2 y+3 z$, 此时有两个约束条件 $g_1=x^2+y^2-$ $2=0$ 与 $g_2=y+z-1=0$. 作Lagrange函数
$$
L(x, y, z, \lambda, \mu)=x+2 y+3 z+\lambda\left(x^2+y^2-2\right)+\mu(y+z-1) .
$$
$
\text{由方程组}\left\{\begin{array}{l}
	L_x=1+2 \lambda x=0 \\
	L_y=2+2 \lambda y+\mu=0 \\
	L_z=3+\mu=0 \\
	L_\lambda=x^2+y^2-2=0 \\
	L_\mu=y+z-1=0
\end{array}\right.
$\\
解得Lagrange函数 $L$ 有两个驻点 $(1,-1,2)$ 和 $(-1,1,0)$.\\
 由于函数最大值和最小 值存在, 故最大值为 $f(1,-1,2)=5$, 最小值为 $f(-1,1,0)=1$.\\
\centerline{$(B)$}
3.5.1. 求曲线 $\left\{\begin{array}{l}z=\sqrt{x} \\ y=0\end{array}\right.$ 与曲线 $\left\{\begin{array}{l}x+2 y-3=0, \\ z=0\end{array}\right.$ 之间的距离.\\
解 :在第一条曲线上任取一点 $(x, 0, \sqrt{x})$, 在第二条曲线上任取一点 $(3-$ $2 v, v, 0)$, 设它们距离的平方为日标函数 $f(x, v)=(x+2 v-3)^2+v^2+x(x \geq 0)$.\\
 由 $\left\{\begin{array}{l}f_x(x, v)=2(x+2 v-3)+1=0 \\ f_v(x, v)=4(x+2 v-3)+2 v=0\end{array}\right.$
  解得唯一驻点 $x=1 / 2, v=1$. \\
  由几何意 义知 $f$ 的最小值存在, 故在两曲线上对应点 $(1 / 2,0,1 / \sqrt{2})$ 与 $(1,1,0)$ 的距离最小, 其值为 $\sqrt{7} / 2$.\\
3.5.2. 设椭球面 $\frac{x^2}{a^2}+\frac{y^2}{b^2}+\frac{z^2}{c^2}=1$ 被通过原点的平面 $l x+m y+n z=0$ 截成一 个椭圆, 求这个椭圆的面积.
解 设目标函数 $f(x, y, z)=x^2+y^2+z^2$, 则考虑函数 $f(x, y, z)$ 在约束条件
$$
\left\{\begin{array}{l}
	\frac{x^2}{a^2}+\frac{y^2}{b^2}+\frac{z^2}{c^2}=1 \\
	l x+m y+n z=0
\end{array}\right.
$$
下的极值. 令
$$
L=x^2+y^2+z^2-\lambda\left(\frac{x^2}{a^2}+\frac{y^2}{b^2}+\frac{z^2}{c^2}-1\right)+\mu(l x+m y+n z) .
$$
由 $\left(L_x, L_y: L_Z\right)=(0,0,0)$ 得到
$$
\begin{cases}L_x=2 x-\frac{2 \lambda x}{a^2}+l \mu=0, & (1) \\ L_y=2 y-\frac{2 \lambda y}{b^2}+m \mu=0, & (2) \\ L_z=2 z-\frac{2 \lambda z}{c^2}+n \mu=0, & (3)\end{cases}
$$
(1) $\times x+(2) \times y+(3) \times z$, 并利用约束条件, 得到 $x^2+y^2+z^2=\lambda$. 联 立 (1), (2), (3)与 $l x+m y+n z=0$, 由于 $(x, y, z, \mu)$ 为方程组的非零解, 所以 系数行列式为 0 , 即
$$
\begin{aligned}
	0 & =\left|\begin{array}{cccc}
		2-\frac{2 \lambda}{a^2} & 0 & 0 & l \\
		0 & 2-\frac{2 \lambda}{b^2} & 0 & m \\
		0 & 0 & 2-\frac{2 \lambda}{c^2} & n \\
		l & m & n & 0
	\end{array}\right| \\
	& =-4 m^2\left(1-\frac{\lambda}{a^2}\right)\left(1-\frac{\lambda}{c^2}\right)-4 n^2\left(1-\frac{\lambda}{a^2}\right)\left(1-\frac{\lambda}{b^2}\right)-4 l^2\left(1-\frac{\lambda}{b^2}\right)\left(1-\frac{\lambda}{c^2}\right) .
\end{aligned}
$$
因此, 有
$$
\left(\frac{m^2}{a^2 c^2}+\frac{n^2}{a^2 b^2}+\frac{l^2}{b^2 c^2}\right) \lambda^2-\left(\frac{a^2+c^2}{a^2 c^2} m^2+\frac{a^2+b^2}{a^2 b^2} n^2+\frac{b^2+c^2}{b^2 c^2} l^2\right) \lambda+m^2+n^2+l^2=0 .
$$\\
得到: $\lambda_1 \lambda_2=\frac{m^2+n^2+l^2}{\frac{m^2}{a^2 c^2}+\frac{n^2}{a^2 b^2}+\frac{l^2}{b^2 c^2}}$, 故所求面积
$$
S=\pi \sqrt{\lambda_1 \lambda_2}=\pi \sqrt{\frac{m^2+n^2+l^2}{\frac{m^2}{a^2 c^2}+\frac{n^2}{a^2 b^2}+\frac{l^2}{b^2 c^2}}}=\pi a b c \sqrt{\frac{m^2+n^2+l^2}{b^2 m^2+c^2 n^2+a^2 l^2}} .
$$
3.5.5. 设函数 $f(x)$ 在 $[1+\infty)$ 内有二阶连续导数, $f(1)=0, f^{\prime}(1)=1$ 且 $z=$ $\left(x^2+y^2\right) f\left(x^2+y^2\right)$ 满足 $\frac{\partial^2 z}{\partial x^2}+\frac{\partial^2 z}{\partial y^2}=0$, 求 $f(x)$ 在 $[1+\infty)$ 上的最大值.\\
解 $z_x=2 x f+2 x\left(x^2+y^2\right) f^{\prime}$, 于是
$$
z_{x x}=2 f+4 x^2 f^{\prime}+2\left(x^2+y^2\right) f^{\prime}+4 x^2 f^{\prime}+4 x^2\left(x^2+y^2\right) f^{\prime \prime}=2 f+2\left(5 x^2+y^{\prime} f^{\prime}+4 x^2\left(x^2+y^2\right) f^{\prime \prime}\right.
$$\\
同理可得 $z_{y y}=2 f+2\left(x^2+5 y^2\right) f^{\prime}+4 y^2\left(x^2+y^2\right) f^{\prime \prime}$. 因此
$$
0=\frac{\partial^2 z}{\partial x^2}+\frac{\partial^2 z}{\partial y^2}=4\left(x^2+y^2\right)^2 f^{\prime \prime}\left(x^2+y^2\right)+12\left(x^2+y^2\right) f^{\prime}\left(x^2+y^2\right)+4 f\left(x^2+y^2\right) \text {. }
$$
若记 $t=x^2+y^2$, 则得到Euler方程
$$
t^2 f^{\prime \prime}(t)+3 t f^{\prime}(t)+f(t)=0
$$
其对应的常系数方程为 $z^{\prime \prime}+2 z^{\prime}+z=0$ (令 $s=\ln t$, 则 $\left.z=f\left(e^s\right)\right)$, 通解为 $z=$ $\left(C_1+C_2 s\right) e^{-s}$. 故, $f(t)=\left(C_1+C_2 \ln t\right) t^{-1}$.\\
 由初值.条件解得 $C_1=0, C_2=1$. 因 此 $f(t)=\frac{\ln t}{t}$, 有唯一驻点 $t=e$, 且 $1<t<e$ 时 $f^{\prime}(t)>0 ; t>e$ 时 $f^{\prime}(t)<0$, 从 而知 $f(e)=e^{-1}$ 为极大值.\\
  又 $\lim _{t \rightarrow+\infty} f(t)=0, f(1)=0$, 故 $f(e)=e^{-1}$ 是 $f(t)$ 在 $[1+$ $\infty)$ 上的最大值.
\section{多元函数微分学在几何上的简单应用}
\centerline{$(A)$}
\noindent3.6.1. 求下列曲线在给定点的切线和法平面方程:
(1) $\vec{r}=\left(t, 2 t^2, t^2\right)$, 在 $t=1$ 处; (2) $\vec{r}=(3 \cos \theta, 3 \sin \theta, 4 \theta)$, 在点 $\left(\frac{3}{\sqrt{2}}, \frac{3}{\sqrt{2}}, \pi\right)$ 处.\\
解 (1) 切线方程: $\frac{x-1}{1}=\frac{y-2}{4}=\frac{z-1}{2}$; 法平面方程: $x+4 y+2 z=11$.
(2) 切线方程: $\frac{x-\frac{3}{\sqrt{2}}}{-3}=\frac{y-\frac{3}{\sqrt{2}}}{3}=\frac{z-\pi}{4 \sqrt{2}}$;\\
法平面方程: $3 x-3 y-4 \sqrt{2} z=-4 \pi \sqrt{2}$ (或 $\frac{3}{\sqrt{2}} x-\frac{3}{\sqrt{2}} y-4 z=-4 \pi$ ).\\
3.6.2. 求下列平面曲线的弧长:
(1) $x^{\frac{2}{3}}+y^{\frac{2}{3}}=a^{\frac{2}{3}},(a>0)$ 的全长; $\quad$ (2) $\rho=a(1+\cos \theta)$ 的全长.
解 (1) $6 a ;$ (2) $8 a$.\\
3.6.3. 求下列空间曲线的弧长:
(1) $\vec{r}=\left(e^t \cos t, e^t \sin t, e^t\right)$ 介于点 $(1,0,1)$ 与点 $\left(0, e^{\frac{\pi}{2}}, e^{\frac{\pi}{2}}\right)$ 之间的弧长;
(3) $\left\{\begin{array}{l}x^2=3 y, \\ 2 x y=9 z\end{array}\right.$ 介于点 $(0,0,0)$ 与点 $(3,3,2)$ 之间的弧长.\\
解 (1) 弧长为: $\sqrt{3} \int_0^{\frac{\pi}{2}} e^t \mathrm{~d} t=\sqrt{3}\left(e^{\frac{\pi}{2}}-1\right)$.
(3) 曲线以 $x$ 为参数, 得到 $y=\frac{x^2}{3}, z=\frac{2 x^3}{27}$. 所以弧长为:
$$
\int_0^3 \sqrt{1+\left(\frac{2}{3} x\right)^2+\left(\frac{2}{9} x^2\right)^2} d x=\int_0^3\left(1+\frac{2}{9} x^2\right) \mathrm{d} x=5
$$\
3.6.4. 求下列曲面在给定点的切平面与法线方程:
(2) $z^2=\frac{x^2}{4}+\frac{y^2}{9}$ 在 $(6,12,5)$ 处; (3) $x^3+y^3+z^3+x y z-6=0$ 在 $(1,2,-1)$ 处.
解 对曲面 $F(x, y, z)=0$, 法向量 $\vec{n}=\left.\left(F_x, F_y, F_z\right)\right|_{p_0}$ 。
(2) 令 $F(x, y, z)=\frac{x^2}{4}+\frac{y^2}{9}-z^2$, 得法向量 $\vec{n}=(3,8 / 3,-10)$.
切平面方程: $9 x+8 y-30 z=0$; 法线方程: $\frac{x-6}{9}=\frac{y-12}{8}=\frac{z-5}{-30}$.
(3) 令 $F(x, y, z)=x^3+y^3+z^3+x y z-6$, 法向量 $\vec{n}=(1,11,5)$.
切平面方程: $x+11 y+5 z=18$; 法线方程: $\frac{x-1}{1}=\frac{y-2}{11}=\frac{z+1}{5}$.\\
3.6.6. (1) 求曲面 $x^2+y^2+z^2=x$ 的切平面, 使它垂直于平面 $x-y-\frac{1}{2} z=2$ 和 切平面, 求此切平面的方程.\\
解 (1) 依题意, 所求切平面的法向量为 $(1,-1,-1 / 2) \times(1,-1,-1)=$ $(1 / 2,1 / 2,0)$, 取 $\vec{n}=(1,1,0)$. 另一方面, 曲面 $x^2+y^2+z^2=x$ 上点 $\left(x_0, y_0, z_0\right)$ 处的法向量为 $\left(2 x_0-1,2 y_0, 2 z_0\right)$, 从而有 $2 x_0-1=2 y_0, z_0=0$, 将其代入曲面方 程 $x^2+y^2+z^2=x$, 得到 $x_0=\frac{2 \pm \sqrt{2}}{4}, y_0=\frac{ \pm \sqrt{2}}{4}, z_0=0$. 故所求切平面为 $\left(x-\frac{2+\sqrt{2}}{4}\right)+\left(y-\frac{\sqrt{2}}{4}\right)=0 \quad$ 和 $\quad\left(x-\frac{2-\sqrt{2}}{4}\right)+\left(y+\frac{\sqrt{2}}{4}\right)=0$, 即 $x+y=\frac{1}{2}(1+\sqrt{2})$ 和 $x+y=\frac{1}{2}(1-\sqrt{2})$.
(2) $9 x+y-z=27$ 与 $9 x+17 y-17 z=-27$.
解 依题意, 所求法线方向向量为 $(1,3,1) \times(1,1,0)=(-1,1,-2)$, 取 $\vec{n}=$ $(1,-1,2)$. 另一方面, 曲面 $x^2+2 y^2+z^2=22$ 上点 $\left(x_0, y_0, z_0\right)$ 处的法向量为 $\left(x_0, 2 y_0, z_0\right)$, 所以 $x_0=-2 y_0=z_0 / 2$. 将其代入曲面方程 $x^2+2 y^2+z^2=22$, 得到 $x_0= \pm 2, y_0=$ $\mp 1, z_0= \pm 4$. 故所求法线为
$$
\frac{x \pm 2}{1}=\frac{y \mp 1}{-1}=\frac{z \pm 4}{2}
$$\\
3.6.9. 求旋转抛物面 $S: z=x^2+y^2$ 和平面 $\pi: x+y-2 z=2$ 平行的切平面 的方程.
解 设 $S$ 上点 $P_0\left(x_0, y_0, z_0\right)$ 处的切平面与平面 $\pi$ 平行. 由于 $S$ 上点 $P_0$ 处法向量 为
$$
\left.\vec{n}\right|_{P_0}=\left(2 x_0, 2 y_0,-1\right),
$$
平面 $\pi$ 的法向量为 $(1,1,-2)$, 按照平面平行的条件, 应该有
$$
\frac{2 x_0}{1}=\frac{2 y_0}{1}=\frac{-1}{-2}=\frac{1}{2},
$$
从而求得 $P_0(1 / 4,1 / 4,1 / 8)$. 因此, 所求切平面方程是 $(x-1 / 4)+(y-1 / 4)-2(z-1 / 8)=0$, 或 $x+y-2 z=1 / 4$.\\
\centerline{$(B)$}
3.6.3. 设函数 $f(u, v)$ 在全平面上有连续的偏导数, 取 $S$ 由方程 $f\left(\frac{x-a}{z-c}, \frac{y-b}{z-c}\right)=$ 0 确定.\\
证明: 该曲面的所有切平面都过点 $(a, b, c)$.
证 记 $F(x, y, z)=f\left(\frac{x-a}{z-c}, \frac{y-b}{z-c}\right)$, 则
$$
\left(F_x, F_y, F_z\right)=\left(\frac{f_1}{z-c}, \frac{f_2}{z-c},-\frac{(x-a) f_1+(y-b) f_2}{(z-c)^2}\right) .
$$
取曲面 $S$ 的法向量
$$
\vec{n}=\left((z-c) f_1,(z-c) f_2,-(x-a) f_1-(y-b) f_2\right) .
$$
记 $(x, y, z)$ 为曲面 $S$ 上的点, $(X, Y, Z)$ 为切平面上的点, 则曲面 $S$ 上过点 $x, y, z$ 的 切平面为
$$
(z-c) f_1(X-x)+(z-c) f_2(Y-y)-\left[(x-a) f_1+(y-b) f_2\right](Z-z)=0 .
$$
对应任意的 $(x, y, z)(z \neq c),(X, Y, Z)=(a, b, c)$ 都满足切平面方程. 证.毕.
\section{空间的曲率}
\centerline{$(A)$}
3.7.1. 求下列平面曲线在给定点的曲率: (2) $y=\sin x$, 在点 $\left(\frac{\pi}{2}, 1\right)$ 处.
解 $\kappa=\left.\frac{|\sin x|}{\left(1+\cos ^2 x\right)^{3 / 2}}\right|_{\frac{\pi}{2}}=1$.\\
3.7.2. 求下列平面曲线的曲率:
(1) $y=a x^2$;
(3) $\vec{r}=(a \cosh t, a \sinh t)$.
解 (1) $\kappa=\frac{2|a|}{\left(1+4 a^2 x^2\right)^{3 / 2}}$. (3) $\kappa=\frac{1}{a(\cosh (2 t))^{3 / 2}}$.
3.7.3. 求下列曲线的曲率 $(a>0)$ :
(1) $\vec{r}=(a \cosh t, a \sinh t, b t) ; \quad$ (3) $\vec{r}=(a(1-\sin t), a(1-\cos t), b t)$.
解 (1) $\kappa=\frac{a \sqrt{b^2 \cosh (2 t)+a^2}}{\left(a^2 \cosh (2 t)+b^2\right)^{3 / 2}}$.
(2) $\kappa=\frac{1}{a^2\left(a^2+b^2\right)}$.\\
3.7.4. 曲线 $y=\ln x$ 上哪一点处的曲率半径最小? 求出该点处的曲率半径.
解 曲率 $\kappa(x)=\frac{x}{\left(1+x^2\right)^{3 / 2}}$. 由 $\kappa^{\prime}(x)=0$ 得到驻点 $x_0=1 / \sqrt{2}$, 经检验它也 是 $\kappa(x)$ 的最大值点. 因此, 曲线 $y=\ln x$ 上点 $(1 / \sqrt{2}, \ln (1 / \sqrt{2}))$ 处的曲率半径最小, 该点处的曲率半径为 $1 / \kappa\left(x_0\right)=3 \sqrt{3} / 2$.

\centerline{$(B)$}
3.7.1. 求曲率 $\kappa(s)=\frac{a}{a^2+s^2}$ 的平面曲线. ( $s$ 是弧长参数)
解
$$
\begin{gathered}
	\kappa(s)=\frac{d \theta}{\mathrm{d} s} \Rightarrow d \theta=\kappa(s) \mathrm{d} s=\frac{a}{a^2+s^2} \mathrm{~d} s, \\
	\theta(s)=\int \kappa(s) \mathrm{d} s=\int \frac{a}{a^2+s^2} \mathrm{~d} s=\arctan \frac{s}{a}+C .
\end{gathered}
$$
设曲线的参数方程为
$$
\left\{\begin{array}{l}
	x=x(s) \\
	y=y(s),
\end{array} s \in[0, l]\right.
$$
由弧微分公式
$$
\mathrm{d} s=\sqrt{(\mathrm{d} x)^2+(\mathrm{d} y)^2}
$$
得
$$
\mathrm{d} x=\cos \theta \mathrm{d} s, \mathrm{~d} y=\sin \theta \mathrm{d} s .
$$
不防设
$$
x(0)=0, y(0)=0, \theta(0)=0,
$$
则
$$
\begin{aligned}
	x(s) & =x(0)+\int_0^s \cos (\theta(s)) \mathrm{d} s=\int_0^s \cos \left(\arctan \frac{s}{a}\right) \mathrm{d} s \\
	& =\int_0^s \frac{a}{\sqrt{a^2+s^2}} \mathrm{~d} s=a\left[\ln \left(s+\sqrt{a^2+s^2}\right)-\ln a\right], \\
	y(s) & =y(0)+\int_0^s \sin (\theta(s)) \mathrm{d} s=\int_0^s \sin \left(\arctan \frac{s}{a}\right) \mathrm{d} s \\
	& =\int_0^s \frac{s}{\sqrt{a^2+s^2}} \mathrm{~d} s=\sqrt{a^2+s^2-a,}
\end{aligned}
$$所以曲线的方程为
$$
\mathbf{r}(s)=\left(a \ln \left(s+\sqrt{a^2+s^2}\right)-a \ln a, \sqrt{a^2+s^2}-a\right) .
$$
消去 $s$, 可得: $y=a\left(\cosh \frac{x}{a}-1\right)$. 事实上, 由 $x=a \ln \left(s+\sqrt{a^2+s^2}\right)-a \ln a$ 知
$$
\frac{x}{a}=\ln \frac{s+\sqrt{a^2+s^2}}{a}=\ln \frac{a}{\sqrt{a^2+s^2}-s},
$$
所以
$$
e^{\frac{x}{a}}=\frac{s+\sqrt{a^2+s^2}}{a}, \quad e^{-\frac{z}{a}}=\frac{\sqrt{a^2+s^2}-s}{a} .
$$
于是, 注意到 $y=\sqrt{a^2+s^2}-a$, 我们有
$$
\frac{e^{\frac{x}{a}}+e^{-\frac{x}{a}}}{2}=\frac{\sqrt{a^2+s^2}}{a} \Rightarrow y=a \cosh \frac{x}{a}-a
$$
3.7.3. 设 $\vec{r}(t)$ 是空间曲线, 曲率为 $\kappa(t)$. 求曲线 $\vec{r}=\vec{r}(-t)$ 的曲率.
解 因为 $\vec{r}^{\prime}(t)=-\vec{r}^{\prime}(-t), \vec{r}^{\prime \prime}(t)=\vec{r}^{\prime \prime}(-t), \vec{r}^{\prime \prime \prime}(t)=-\vec{r}^{\prime \prime \prime}(-t)$, 所以曲线 $\vec{r}=$ $\vec{r}(-t)$ 的曲率为
$$
\tilde{\kappa}(t)=\frac{\left\|-\vec{r}^{\prime}(-t) \times \vec{r}^{\prime \prime}(-t)\right\|}{\left\|-\vec{r}^{\prime}(-t)\right\|^3}=\frac{\left\|\vec{r}^{\prime}(-t) \times \vec{r}^{\prime \prime}(-t)\right\|}{\left\|\vec{r}^{\prime}(-t)\right\|^3}=\kappa(-t) .
$$\\
\section{多元向量值函数的导数和微分}
\centerline{$(A)$}
3.8.1. 求下列向量值函数的Jacobi矩阵:
(1) $\vec{f}(x, y)=\left(x^2+\sin y, 2 x y\right)^T$
(3) $\vec{f}(x, y, z)=\left(x \cos y, y e^x: \sin (x z)\right)^T$.
解 (1) $\mathrm{D} \vec{f}=\left(\begin{array}{cc}2 x & \cos y \\ 2 y & 2 x\end{array}\right)$. (3) $\mathrm{D} \vec{f}=\left(\begin{array}{ccc}\cos y & -x \sin y & 0 \\ y e^x & e^x & 0 \\ z \cos (x z) & 0 & x \cos (x z)\end{array}\right)$.
3.8.3. 求向量值函数 $\vec{f}(x, y)=\left(\arctan x, e^{x y}\right)^T$ 的导数 $D \vec{f}(x, y)$.
解 $\mathrm{D} \vec{f}=\left(\begin{array}{cc}\frac{1}{1+x^2} & 0 \\ y e^{x y} & x e^{x y}\end{array}\right)$.\\
3.8.6. 设向量值函数 $\vec{f}: \mathbb{R}^3 \rightarrow \mathbb{R}^2$ 定义 $\vec{f}(x, y, z)=\left(e^x \cos y+e^y z^2, 2 x \sin y-\right.$ $\left.3 y z^3\right)^T$, 求 $\mathrm{D} \vec{f}\left(0, \frac{\pi}{2}, 1\right)$.
解 原式 $=\left(\begin{array}{ccc}e^x \cos y & -e^x \sin y+e^y z^2 & 2 z e^y \\ 2 \sin y & 2 x \cos y-3 z^3 & -9 y z^2\end{array}\right)_{\left(0, \frac{\pi}{2}, 1\right)}=\left(\begin{array}{ccc}0 & -1+e^{\frac{\pi}{2}} & 2 e^{\frac{\pi}{2}} \\ 2 & -3 & -\frac{9}{2} \pi\end{array}\right)$.\\
\centerline{$(B)$}
\newpage
\noindent


\end{document}
